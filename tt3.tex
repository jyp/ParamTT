\documentclass[10pt,a4paper]{article}
\usepackage[utf8]{inputenc}
\usepackage{amssymb,amstext,amsmath}
\usepackage{mathpartir}
\usepackage{mytheorems}
\usepackage[a4paper,margin=1.8cm]{geometry}
\usepackage{fancyref}

\newcommand\CC[4]{(#2,_{#1} #3)}
\newcommand\CP[3]{(#2,_{#1} #3)}
\newcommand\CSig[1]{\Sigma^{#1}}
\newcommand\CTimes[3]{#2 ×_{#1} #3}
\newcommand\SW[2]{\mathsf{SW}^{#1}_{#2}}
\newcommand\sw[2]{\mathsf{sw}^{#1}_{#2}}
\newcommand\dom{\mathsf{dom}}
\newcommand\param[1]{\!\cdot\!#1}
\newcommand\pvar[2]{{#1}^{(#2)}}
\newcommand\op[1]{∋_{#1}}
\newcommand\ip{Ψ}


\DeclareUnicodeCharacter{00A0}{~} %   NO-BREAK SPACE
\DeclareUnicodeCharacter{00A7}{\S} % §
\DeclareUnicodeCharacter{00AC}{\ensuremath{\neg}} % ¬
\DeclareUnicodeCharacter{00B0}{^{\circ}} % °
\DeclareUnicodeCharacter{00B1}{^1} 
\DeclareUnicodeCharacter{00B2}{^2} % ²
\DeclareUnicodeCharacter{00B7}{\ensuremath{\cdot}} % ·
\DeclareUnicodeCharacter{00B9}{\textsuperscript{l}} % ¹
\DeclareUnicodeCharacter{00D7}{\ensuremath{\times}} % × 
\DeclareUnicodeCharacter{00F7}{\ensuremath{\div}} % ÷
\DeclareUnicodeCharacter{02E1}{\ensuremath{{^l}}} % ˡ
\DeclareUnicodeCharacter{02B3}{\ensuremath{{^r}}} % ʳ
\DeclareUnicodeCharacter{0393}{\ensuremath{\Gamma}} % Γ
\DeclareUnicodeCharacter{0394}{\ensuremath{\Delta}} % Δ
\DeclareUnicodeCharacter{0397}{\ensuremath{\textrm{H}}} % Η
\DeclareUnicodeCharacter{0398}{\ensuremath{\Theta}} % Θ
\DeclareUnicodeCharacter{039B}{\ensuremath{\Lambda}} % Λ
\DeclareUnicodeCharacter{039E}{\ensuremath{\Xi}} % Ξ
\DeclareUnicodeCharacter{03A3}{\ensuremath{\Sigma}} % Σ
\DeclareUnicodeCharacter{03A6}{\ensuremath{\Phi}} % Φ
\DeclareUnicodeCharacter{03A8}{\ensuremath{\Psi}} % Ψ
\DeclareUnicodeCharacter{03A9}{\ensuremath{\Omega}} % Ω
\DeclareUnicodeCharacter{03B1}{\ensuremath{\mathnormal{\alpha}}} % α
\DeclareUnicodeCharacter{03B2}{\ensuremath{\beta}} % β
\DeclareUnicodeCharacter{03B3}{\ensuremath{\mathnormal{\gamma}}} % γ
\DeclareUnicodeCharacter{03B4}{\ensuremath{\mathnormal{\delta}}} % δ
\DeclareUnicodeCharacter{03B5}{\ensuremath{\mathnormal{\varepsilon}}} % ε
\DeclareUnicodeCharacter{03B6}{\ensuremath{\mathnormal{\zeta}}} % ζ
\DeclareUnicodeCharacter{03B7}{\ensuremath{\mathnormal{\eta}}} % η
\DeclareUnicodeCharacter{03B8}{\ensuremath{\mathnormal{\theta}}} % θ
\DeclareUnicodeCharacter{03B9}{\ensuremath{\mathnormal{\iota}}} % ι
\DeclareUnicodeCharacter{03BA}{\ensuremath{\mathnormal{\kappa}}} % κ
\DeclareUnicodeCharacter{03BB}{\ensuremath{\mathnormal{\lambda}}} % λ
\DeclareUnicodeCharacter{03BC}{\ensuremath{\mathnormal{\mu}}} % μ
\DeclareUnicodeCharacter{03BD}{\ensuremath{\mathnormal{\mu}}} % ν
\DeclareUnicodeCharacter{03BE}{\ensuremath{\mathnormal{\xi}}} % ξ
\DeclareUnicodeCharacter{03C0}{\ensuremath{\mathnormal{\pi}}} % π
\DeclareUnicodeCharacter{03C1}{\ensuremath{\mathnormal{\rho}}} % ρ
\DeclareUnicodeCharacter{03C3}{\ensuremath{\mathnormal{\sigma}}} % σ
\DeclareUnicodeCharacter{03C4}{\ensuremath{\mathnormal{\tau}}} % τ
\DeclareUnicodeCharacter{03C6}{\ensuremath{\mathnormal{\varphi}}} % φ
\DeclareUnicodeCharacter{03D5}{\ensuremath{\mathnormal{\phi}}} % ϕ
\DeclareUnicodeCharacter{03C7}{\ensuremath{\mathnormal{\chi}}} % χ
\DeclareUnicodeCharacter{03C8}{\ensuremath{\mathnormal{\psi}}} % ψ
\DeclareUnicodeCharacter{03C9}{\ensuremath{\mathnormal{\omega}}} % ω 
\DeclareUnicodeCharacter{03F5}{\ensuremath{\mathnormal{\epsilon}}} % ϵ
\DeclareUnicodeCharacter{1D62}{_i} % ᵢ
\DeclareUnicodeCharacter{10627}{\ensuremath{\lbana}} 
\DeclareUnicodeCharacter{10628}{\ensuremath{\rbana}} 
\DeclareUnicodeCharacter{2026}{\ensuremath{\ldots}}
\DeclareUnicodeCharacter{202F}{{\,}}
\DeclareUnicodeCharacter{2080}{\ensuremath{_0}} % ₀
\DeclareUnicodeCharacter{2081}{\ensuremath{_1}}
\DeclareUnicodeCharacter{2082}{\ensuremath{_2}}
\DeclareUnicodeCharacter{2083}{\ensuremath{_3}}
\DeclareUnicodeCharacter{2084}{\ensuremath{_4}}
\DeclareUnicodeCharacter{2085}{\ensuremath{_5}}
\DeclareUnicodeCharacter{2086}{\ensuremath{_6}}
\DeclareUnicodeCharacter{2087}{\ensuremath{_7}}
\DeclareUnicodeCharacter{2088}{\ensuremath{_8}}
\DeclareUnicodeCharacter{2089}{\ensuremath{_9}}
\DeclareUnicodeCharacter{2115}{\mathbb{N}}
\DeclareUnicodeCharacter{214B}{\ensuremath{\parr}}
\DeclareUnicodeCharacter{2190}{\ensuremath{\leftarrow}} % ← 
\DeclareUnicodeCharacter{2191}{\ensuremath{\uparrow}} % ↑
\DeclareUnicodeCharacter{2192}{\ensuremath{\rightarrow}} % →
\DeclareUnicodeCharacter{2194}{\ensuremath{\leftrightarrow}} % ↔
\DeclareUnicodeCharacter{2196}{\nwarrow} % ↖
\DeclareUnicodeCharacter{2197}{\nearrow} % ↗
\DeclareUnicodeCharacter{219D}{\ensuremath{\leadsto}} % ↝
\DeclareUnicodeCharacter{21A6}{\ensuremath{\mapsto}} % ↦ 
\DeclareUnicodeCharacter{21C6}{\ensuremath{\leftrightarrows}} % ⇆
\DeclareUnicodeCharacter{21D0}{\ensuremath{\Leftarrow}} % ⇐
\DeclareUnicodeCharacter{21D2}{\ensuremath{\Rightarrow}} % ⇒ 
\DeclareUnicodeCharacter{21D4}{\ensuremath{\Leftrightarrow}} % ⇔
\DeclareUnicodeCharacter{2200}{\ensuremath{\forall}} % ∀
\DeclareUnicodeCharacter{2203}{\ensuremath{\exists}} % ∃
\DeclareUnicodeCharacter{2205}{\ensuremath{\varnothing}} % ∅
\DeclareUnicodeCharacter{2208}{\ensuremath{\in}} % ∈
\DeclareUnicodeCharacter{2209}{\ensuremath{\not\in}} % ∉
\DeclareUnicodeCharacter{220B}{\ensuremath{\ni}}
\DeclareUnicodeCharacter{220E}{\ensuremath{\qed}} % ∎ % Alternatively use \blacksquare
\DeclareUnicodeCharacter{2211}{\sum}% ∑
\DeclareUnicodeCharacter{2215}{\mathbb{N}} % ℕ
\DeclareUnicodeCharacter{2217}{\ensuremath{\ast}} % ∗
\DeclareUnicodeCharacter{2218}{\ensuremath{\circ}} % ∘
\DeclareUnicodeCharacter{2219}{\ensuremath{\bullet}} % ∙ 
\DeclareUnicodeCharacter{221E}{\ensuremath{\infty}} % ∞
\DeclareUnicodeCharacter{2223}{\ensuremath{\mid}} % ∣
\DeclareUnicodeCharacter{2227}{\wedge}% ∧
\DeclareUnicodeCharacter{2228}{\vee}% ∨
\DeclareUnicodeCharacter{2229}{\ensuremath{\cap}} % ∩
\DeclareUnicodeCharacter{222A}{\ensuremath{\cup}} % ∪
\DeclareUnicodeCharacter{2237}{::} % ∷
\DeclareUnicodeCharacter{223C}{\ensuremath{\sim}} % ∼
\DeclareUnicodeCharacter{2243}{\ensuremath{\simeq}} % ≃
\DeclareUnicodeCharacter{2245}{\ensuremath{\cong}} % ≅ 
\DeclareUnicodeCharacter{2248}{\ensuremath{\approx}} % ≈
\DeclareUnicodeCharacter{225C}{\ensuremath{\stackrel{\scriptscriptstyle {\triangle}}{=}}} % ≜
\DeclareUnicodeCharacter{225F}{\ensuremath{\stackrel{\scriptscriptstyle ?}{=}}} % ≟
\DeclareUnicodeCharacter{2260}{\neq}% ≠
\DeclareUnicodeCharacter{2261}{\equiv}% ≡
\DeclareUnicodeCharacter{2264}{\ensuremath{\le}} % ≤
\DeclareUnicodeCharacter{2265}{\ensuremath{\ge}} % ≥
\DeclareUnicodeCharacter{2282}{\ensuremath{\subset}} % ⊂
\DeclareUnicodeCharacter{2283}{\ensuremath{\supset}} % ⊃ 
\DeclareUnicodeCharacter{2286}{\ensuremath{\subseteq}} % ⊆ 
\DeclareUnicodeCharacter{2287}{\ensuremath{\supseteq}} % ⊇
\DeclareUnicodeCharacter{2293}{\ensuremath{\sqcup}} % ⊓
\DeclareUnicodeCharacter{2293}{\sqcap} % ⊓
\DeclareUnicodeCharacter{2294}{\sqcup} % ⊔
\DeclareUnicodeCharacter{2295}{\ensuremath{\oplus}} % ⊕
\DeclareUnicodeCharacter{2297}{\ensuremath{\otimes}} % ⊗
\DeclareUnicodeCharacter{22A2}{\ensuremath{\vdash}}
\DeclareUnicodeCharacter{22A4}{\ensuremath{\top}} % ⊤
\DeclareUnicodeCharacter{22A5}{\ensuremath{\bot}} % ⊥
\DeclareUnicodeCharacter{22A7}{\models} % ⊧ 
\DeclareUnicodeCharacter{22A8}{\models} % ⊨
\DeclareUnicodeCharacter{22A9}{\Vdash} % ⊩
\DeclareUnicodeCharacter{22B8}{\ensuremath{\multimap}} % ⊸
\DeclareUnicodeCharacter{22C4}{\diamond} % ⋄
\DeclareUnicodeCharacter{22C6}{\ensuremath{\star}}
\DeclareUnicodeCharacter{22EE}{\ensuremath{\vdots}} % ⋮
\DeclareUnicodeCharacter{22EF}{\ensuremath{\cdots}} % ⋯
\DeclareUnicodeCharacter{2308}{\ensuremath{\lceil}}
\DeclareUnicodeCharacter{2309}{\ensuremath{\rceil}}
\DeclareUnicodeCharacter{230A}{\ensuremath{\lfloor}}
\DeclareUnicodeCharacter{230B}{\ensuremath{\rfloor}}
\DeclareUnicodeCharacter{25A1}{\ensuremath{\square}} % □
\DeclareUnicodeCharacter{25AF}{\mathop{\talloblong}} % ▯
\DeclareUnicodeCharacter{25C7}{\diamond} % ◇
\DeclareUnicodeCharacter{2605}{\ensuremath{\star}}   % ★
\DeclareUnicodeCharacter{2713}{\ensuremath{\checkmark}} % ✓
\DeclareUnicodeCharacter{27C2}{\ensuremath{^\bot}} % PERPENDICULAR ⟂
\DeclareUnicodeCharacter{27E6}{\ensuremath{\llbracket}} % ⟦
\DeclareUnicodeCharacter{27E7}{\ensuremath{\rrbracket}} % ⟧
\DeclareUnicodeCharacter{27E8}{\ensuremath{\langle}} % ⟨
\DeclareUnicodeCharacter{27E9}{\ensuremath{\rangle}} % ⟩
\DeclareUnicodeCharacter{27F6}{{\longrightarrow}} % ⟶
\DeclareUnicodeCharacter{27F7}{{\longleftrightarrow}} % ⟷
\DeclareUnicodeCharacter{2A04}{\mathop{\dot{\cup}}} % ⨄
\DeclareUnicodeCharacter{2AFE}{\mathop{\talloblong}} % ⫾

% \DeclareUnicodeCharacter{8499}{\mathcal{M}} 
% \DeclareUnicodeCharacter{8718}{\ensuremath{\blacksquare}}
% \DeclareUnicodeCharacter{8797}{\mathrel{\mathop:}=}
% \DeclareUnicodeCharacter{9657}{\ensuremath{\triangleright}}
% \DeclareUnicodeCharacter{9667}{\triangleright{}}
% \DeclareUnicodeCharacter{9669}{\ensuremath{\triangleleft}}


\title{A parametric Type theory}

\author{}
\date{\today}

\begin{document}
\maketitle


\begin{definition}[Syntax of terms and contexts]
  
  \begin{align*}
    t,u,A,B & ::= x ~|~ U ~|~ λx:A. t      ~|~ t u ~|~ (x:A) → B \\
            & ~|~ \CC i t u A  ~|~ t \param i  \\
            & ~|~ A ∋ u ~|~ \ip A \\
    \Gamma,\Delta & ::= () ~|~ \Gamma,x:A ~|~ \Gamma,I
  \end{align*}
\end{definition}
$\CSig i A P$ and $(x:A) ×_i P x$ are other notations for $(A ,_i \ip P)$.

$A \op i x$ is a notation for $A \param i ∋ x$.

We use $I,J,…$ for finite {\em sets} of colours (order does not matter).

\begin{definition}[Typing relation]
Contexts:
  \begin{mathpar}
    \inferrule[Empty]{~}{() ⊢ }

    \inferrule[NewVar]{Γ⊢ \\ Γ ⊢ A }{ Γ,x:A ⊢ }

    \inferrule[NewColor]{Γ⊢}{ Γ,I ⊢ }

  \end{mathpar}
 
Types:
 \begin{mathpar}
    \inferrule[Universe]{Γ ⊢}{Γ ⊢ U : U}

    \inferrule[Pi]{Γ ⊢ A : U \\ Γ,x:A ⊢ B : U}{Γ ⊢ (x:A) → B : U}

    \inferrule[Out-Pred]{Γ ⊢ P : U \op i A \\ Γ ⊢ a : A}{Γ ⊢ P ∋ a : U}

    \inferrule[In-Pred]{Γ ⊢ P : A → U}{Γ ⊢ \ip P : U \op i A}
 \end{mathpar}

Terms:
  \begin{mathpar}
    \inferrule[Conv]{Γ ⊢ t:A \\ A = B}{Γ ⊢ t : B}

    \inferrule[Var]{Γ ⊢ A}{Γ,x:A ⊢ x : A}

    \inferrule[Wk]{Γ ⊢ A \\Γ ⊢ t : T}{Γ,x:A ⊢ t : T}

    \inferrule[CWk]{Γ ⊢ t : T}{Γ,I ⊢ t : T}

    \\\inferrule[Lam]{Γ,x:A ⊢ B}{Γ ⊢ λx:A.b : (x:A) → B}
 
    \inferrule[App]{Γ ⊢ t : (x:A) → B[x] \\ Γ ⊢ u : A}{Γ ⊢ t u: B[u] }

    \\\inferrule[Color-intro]{Γ,I ⊢ a : T i0 \\Γ,I ⊢ p : T \param i a}{Γ,I,i ⊢ \CC i a p T : T}

    \inferrule[Color-elim]{Γ,I,i ⊢ a : T}{Γ,I ⊢ a \param i : T \op i  a (i0)}


  \end{mathpar}
\end{definition}

\begin{definition}[Reduction]~

Functions:
\begin{align*}
  (λx:A. u[x]) t &= u[t]  \\
  (f ,_i g)_{(x:A)→ B[x]} a & = (f a(i0) ,_i g a(i0) a \param i)_{B[a]}
\end{align*}

Param:
\begin{align*}
  \ip P ∋ a &= P a \\
  ((x:A) → B[x]) \op i f &= (x:A) → (x' : A\op i x) → (B[(x,_i x')_A])\op i (f x)  & ??? \\
  (λx:A. b[x])\param i &= λx:A(i0). λx':A\op i x. b[\CC i x {x'} A]\param i \\
  (f a)\param i &= f\param i (a (i0)) (a\param i) \\
  (a,_i p) \param i  &= p \\
  (a,_i p) \param j  &= (a \param j ,_i p \param j) \\
\end{align*}
\end{definition}

\begin{definition}[Conversion]
  \begin{mathpar}
    \inferrule{t x = u} {t = λ x:A.u}

    \inferrule{}{(T \op i {\CP j a p}) \op j q = (T \op j {\CP i a q}) \op i p}

    \inferrule{t (i0) = a \\ t.i = p} {t = \CC i a p {\CSig i A P}}

  \end{mathpar}
+ congruences.
\end{definition}

\begin{definition}
 We define substitution between contexts

  \begin{align*}
    σ, δ & ::= ()  |  (σ,x=t)  |  (σ,f)
  \end{align*}

  \begin{mathpar}
    \inferrule {σ : Δ → Γ\\ Δ ⊢ t:Aσ} {(σ,x=t) : Δ → Γ,x:A} 

    \inferrule {σ : Δ → Γ} {(σ,f) : Δ → Γ,J}
  \end{mathpar}
where $f$ is a function of domain $J$ and taking as values $0,1$ or
colours declared in $Δ$ (and such that $j=k$ if $f(j) = f(k)$ is a colour).

The identity substitution $1_{Γ}$ is defined as usual and $(i0)$ is
an abbreviation for $(1_{Γ},i=0)$. Also $b[u]$ is an abbreviation for
$b(1_{Γ},x=u)$.
\end{definition}

\begin{definition}
We define $tσ$ by induction on $t$

\begin{align*}
  Uσ &= U \\
  x(σ,f) &= xσ \\
  x(σ,y=u) &= xσ \\
  x(σ,x=u) &= u \\
  ((x:A)→ B)σ &= (y:Aσ) → B(σ,x=y) \\
  (λ x:A. b)σ &= λ y:Aσ. b(σ,x=y) \\
  (t.i)σ &= (t(σ,i=j)).j \\
  (f a)σ &= fσ  (aσ) \\
  (a,_i b)_T(σ,x=u)  &= (a,_i b)_Tσ  \\
  (a,_i b)_T(σ,f)  &= (a,_i p)_Tσ & i \# f \\
  (a,_i b)_T(σ,f)  &= a(σ,f-i) & f(i) = 0 \\
  (a,_i b)_T(σ,f)  &= (a(σ,f-i),_j b(σ,f-i))_{T(σ,f)} & f(i) = j
\end{align*}
\end{definition}



\subsection{Semantics}

- Nominality; dependence on a finite number of colors.

Le modele correspond a pretendre connaitre toutes les expansions en hypercubes.

un type A sur i,j,... couleurs est

A : Set
A.i : A -> Set
A.j : A -> Set
A.i.j : (x:A) -> A.i x -> A.j x -> Set
...

mais A.j.i est donne par

A.j.i a p q = A.i.j a q p     (eq. 1)

Le swap est donc purement formel, et uniquement dans les types.

En effet, avoir a : A sur i,j,... couleurs est

a : A
a.i : A.i a
a.j : A.j a
a.i.j : A.i.j a a.i a.j
...

a.j.i est donne par a.i.j (pas de transformation)


\end{document}
