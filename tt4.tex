\documentclass[10pt,a4paper]{article}
\usepackage[utf8]{inputenc}
\usepackage{stmaryrd}
\usepackage{amssymb,amstext,amsmath}
\usepackage{mathtools}
\usepackage{mathpartir}
\usepackage{mytheorems}
\usepackage[a4paper,margin=1.8cm]{geometry}
\usepackage{fancyref}

\newcommand\CC[4]{(#2,_{#1} #3)}
\newcommand\CP[3]{(#2,_{#1} #3)}
\newcommand\CSig[1]{\Sigma^{#1}}
\newcommand\CTimes[3]{#2 ×_{#1} #3}
\newcommand\SW[2]{\mathsf{SW}^{#1}_{#2}}
\newcommand\sw[2]{\mathsf{sw}^{#1}_{#2}}
\newcommand\dom{\mathsf{dom}}
\newcommand\param[1]{\!\cdot\!#1}
\newcommand\pvar[2]{{#1}^{(#2)}}
\newcommand\op[1]{∋_{#1}}
\newcommand\ip[3]{Σ^{#1} {#2}\,{#3}}
\newcommand\fp[3]{⟨#2 ,_{#1} #3⟩}
\newcommand\proj[2]{⌊{#2}⌋_{#1}}
\DeclareUnicodeCharacter{00A0}{~} %   NO-BREAK SPACE
\DeclareUnicodeCharacter{00A7}{\S} % §
\DeclareUnicodeCharacter{00AC}{\ensuremath{\neg}} % ¬
\DeclareUnicodeCharacter{00B0}{^{\circ}} % °
\DeclareUnicodeCharacter{00B1}{^1} 
\DeclareUnicodeCharacter{00B2}{^2} % ²
\DeclareUnicodeCharacter{00B7}{\ensuremath{\cdot}} % ·
\DeclareUnicodeCharacter{00B9}{\textsuperscript{l}} % ¹
\DeclareUnicodeCharacter{00D7}{\ensuremath{\times}} % × 
\DeclareUnicodeCharacter{00F7}{\ensuremath{\div}} % ÷
\DeclareUnicodeCharacter{02E1}{\ensuremath{{^l}}} % ˡ
\DeclareUnicodeCharacter{02B3}{\ensuremath{{^r}}} % ʳ
\DeclareUnicodeCharacter{0393}{\ensuremath{\Gamma}} % Γ
\DeclareUnicodeCharacter{0394}{\ensuremath{\Delta}} % Δ
\DeclareUnicodeCharacter{0397}{\ensuremath{\textrm{H}}} % Η
\DeclareUnicodeCharacter{0398}{\ensuremath{\Theta}} % Θ
\DeclareUnicodeCharacter{039B}{\ensuremath{\Lambda}} % Λ
\DeclareUnicodeCharacter{039E}{\ensuremath{\Xi}} % Ξ
\DeclareUnicodeCharacter{03A3}{\ensuremath{\Sigma}} % Σ
\DeclareUnicodeCharacter{03A6}{\ensuremath{\Phi}} % Φ
\DeclareUnicodeCharacter{03A8}{\ensuremath{\Psi}} % Ψ
\DeclareUnicodeCharacter{03A9}{\ensuremath{\Omega}} % Ω
\DeclareUnicodeCharacter{03B1}{\ensuremath{\mathnormal{\alpha}}} % α
\DeclareUnicodeCharacter{03B2}{\ensuremath{\beta}} % β
\DeclareUnicodeCharacter{03B3}{\ensuremath{\mathnormal{\gamma}}} % γ
\DeclareUnicodeCharacter{03B4}{\ensuremath{\mathnormal{\delta}}} % δ
\DeclareUnicodeCharacter{03B5}{\ensuremath{\mathnormal{\varepsilon}}} % ε
\DeclareUnicodeCharacter{03B6}{\ensuremath{\mathnormal{\zeta}}} % ζ
\DeclareUnicodeCharacter{03B7}{\ensuremath{\mathnormal{\eta}}} % η
\DeclareUnicodeCharacter{03B8}{\ensuremath{\mathnormal{\theta}}} % θ
\DeclareUnicodeCharacter{03B9}{\ensuremath{\mathnormal{\iota}}} % ι
\DeclareUnicodeCharacter{03BA}{\ensuremath{\mathnormal{\kappa}}} % κ
\DeclareUnicodeCharacter{03BB}{\ensuremath{\mathnormal{\lambda}}} % λ
\DeclareUnicodeCharacter{03BC}{\ensuremath{\mathnormal{\mu}}} % μ
\DeclareUnicodeCharacter{03BD}{\ensuremath{\mathnormal{\mu}}} % ν
\DeclareUnicodeCharacter{03BE}{\ensuremath{\mathnormal{\xi}}} % ξ
\DeclareUnicodeCharacter{03C0}{\ensuremath{\mathnormal{\pi}}} % π
\DeclareUnicodeCharacter{03C1}{\ensuremath{\mathnormal{\rho}}} % ρ
\DeclareUnicodeCharacter{03C3}{\ensuremath{\mathnormal{\sigma}}} % σ
\DeclareUnicodeCharacter{03C4}{\ensuremath{\mathnormal{\tau}}} % τ
\DeclareUnicodeCharacter{03C6}{\ensuremath{\mathnormal{\varphi}}} % φ
\DeclareUnicodeCharacter{03D5}{\ensuremath{\mathnormal{\phi}}} % ϕ
\DeclareUnicodeCharacter{03C7}{\ensuremath{\mathnormal{\chi}}} % χ
\DeclareUnicodeCharacter{03C8}{\ensuremath{\mathnormal{\psi}}} % ψ
\DeclareUnicodeCharacter{03C9}{\ensuremath{\mathnormal{\omega}}} % ω 
\DeclareUnicodeCharacter{03F5}{\ensuremath{\mathnormal{\epsilon}}} % ϵ
\DeclareUnicodeCharacter{1D62}{_i} % ᵢ
\DeclareUnicodeCharacter{10627}{\ensuremath{\lbana}} 
\DeclareUnicodeCharacter{10628}{\ensuremath{\rbana}} 
\DeclareUnicodeCharacter{2026}{\ensuremath{\ldots}}
\DeclareUnicodeCharacter{202F}{{\,}}
\DeclareUnicodeCharacter{2080}{\ensuremath{_0}} % ₀
\DeclareUnicodeCharacter{2081}{\ensuremath{_1}}
\DeclareUnicodeCharacter{2082}{\ensuremath{_2}}
\DeclareUnicodeCharacter{2083}{\ensuremath{_3}}
\DeclareUnicodeCharacter{2084}{\ensuremath{_4}}
\DeclareUnicodeCharacter{2085}{\ensuremath{_5}}
\DeclareUnicodeCharacter{2086}{\ensuremath{_6}}
\DeclareUnicodeCharacter{2087}{\ensuremath{_7}}
\DeclareUnicodeCharacter{2088}{\ensuremath{_8}}
\DeclareUnicodeCharacter{2089}{\ensuremath{_9}}
\DeclareUnicodeCharacter{2115}{\mathbb{N}}
\DeclareUnicodeCharacter{214B}{\ensuremath{\parr}}
\DeclareUnicodeCharacter{2190}{\ensuremath{\leftarrow}} % ← 
\DeclareUnicodeCharacter{2191}{\ensuremath{\uparrow}} % ↑
\DeclareUnicodeCharacter{2192}{\ensuremath{\rightarrow}} % →
\DeclareUnicodeCharacter{2194}{\ensuremath{\leftrightarrow}} % ↔
\DeclareUnicodeCharacter{2196}{\nwarrow} % ↖
\DeclareUnicodeCharacter{2197}{\nearrow} % ↗
\DeclareUnicodeCharacter{219D}{\ensuremath{\leadsto}} % ↝
\DeclareUnicodeCharacter{21A6}{\ensuremath{\mapsto}} % ↦ 
\DeclareUnicodeCharacter{21C6}{\ensuremath{\leftrightarrows}} % ⇆
\DeclareUnicodeCharacter{21D0}{\ensuremath{\Leftarrow}} % ⇐
\DeclareUnicodeCharacter{21D2}{\ensuremath{\Rightarrow}} % ⇒ 
\DeclareUnicodeCharacter{21D4}{\ensuremath{\Leftrightarrow}} % ⇔
\DeclareUnicodeCharacter{2200}{\ensuremath{\forall}} % ∀
\DeclareUnicodeCharacter{2203}{\ensuremath{\exists}} % ∃
\DeclareUnicodeCharacter{2205}{\ensuremath{\varnothing}} % ∅
\DeclareUnicodeCharacter{2208}{\ensuremath{\in}} % ∈
\DeclareUnicodeCharacter{2209}{\ensuremath{\not\in}} % ∉
\DeclareUnicodeCharacter{220B}{\ensuremath{\ni}}
\DeclareUnicodeCharacter{220E}{\ensuremath{\qed}} % ∎ % Alternatively use \blacksquare
\DeclareUnicodeCharacter{2211}{\sum}% ∑
\DeclareUnicodeCharacter{2215}{\mathbb{N}} % ℕ
\DeclareUnicodeCharacter{2217}{\ensuremath{\ast}} % ∗
\DeclareUnicodeCharacter{2218}{\ensuremath{\circ}} % ∘
\DeclareUnicodeCharacter{2219}{\ensuremath{\bullet}} % ∙ 
\DeclareUnicodeCharacter{221E}{\ensuremath{\infty}} % ∞
\DeclareUnicodeCharacter{2223}{\ensuremath{\mid}} % ∣
\DeclareUnicodeCharacter{2227}{\wedge}% ∧
\DeclareUnicodeCharacter{2228}{\vee}% ∨
\DeclareUnicodeCharacter{2229}{\ensuremath{\cap}} % ∩
\DeclareUnicodeCharacter{222A}{\ensuremath{\cup}} % ∪
\DeclareUnicodeCharacter{2237}{::} % ∷
\DeclareUnicodeCharacter{223C}{\ensuremath{\sim}} % ∼
\DeclareUnicodeCharacter{2243}{\ensuremath{\simeq}} % ≃
\DeclareUnicodeCharacter{2245}{\ensuremath{\cong}} % ≅ 
\DeclareUnicodeCharacter{2248}{\ensuremath{\approx}} % ≈
\DeclareUnicodeCharacter{225C}{\ensuremath{\stackrel{\scriptscriptstyle {\triangle}}{=}}} % ≜
\DeclareUnicodeCharacter{225F}{\ensuremath{\stackrel{\scriptscriptstyle ?}{=}}} % ≟
\DeclareUnicodeCharacter{2260}{\neq}% ≠
\DeclareUnicodeCharacter{2261}{\equiv}% ≡
\DeclareUnicodeCharacter{2264}{\ensuremath{\le}} % ≤
\DeclareUnicodeCharacter{2265}{\ensuremath{\ge}} % ≥
\DeclareUnicodeCharacter{2282}{\ensuremath{\subset}} % ⊂
\DeclareUnicodeCharacter{2283}{\ensuremath{\supset}} % ⊃ 
\DeclareUnicodeCharacter{2286}{\ensuremath{\subseteq}} % ⊆ 
\DeclareUnicodeCharacter{2287}{\ensuremath{\supseteq}} % ⊇
\DeclareUnicodeCharacter{2293}{\ensuremath{\sqcup}} % ⊓
\DeclareUnicodeCharacter{2293}{\sqcap} % ⊓
\DeclareUnicodeCharacter{2294}{\sqcup} % ⊔
\DeclareUnicodeCharacter{2295}{\ensuremath{\oplus}} % ⊕
\DeclareUnicodeCharacter{2297}{\ensuremath{\otimes}} % ⊗
\DeclareUnicodeCharacter{22A2}{\ensuremath{\vdash}}
\DeclareUnicodeCharacter{22A4}{\ensuremath{\top}} % ⊤
\DeclareUnicodeCharacter{22A5}{\ensuremath{\bot}} % ⊥
\DeclareUnicodeCharacter{22A7}{\models} % ⊧ 
\DeclareUnicodeCharacter{22A8}{\models} % ⊨
\DeclareUnicodeCharacter{22A9}{\Vdash} % ⊩
\DeclareUnicodeCharacter{22B8}{\ensuremath{\multimap}} % ⊸
\DeclareUnicodeCharacter{22C4}{\diamond} % ⋄
\DeclareUnicodeCharacter{22C6}{\ensuremath{\star}}
\DeclareUnicodeCharacter{22EE}{\ensuremath{\vdots}} % ⋮
\DeclareUnicodeCharacter{22EF}{\ensuremath{\cdots}} % ⋯
\DeclareUnicodeCharacter{2308}{\ensuremath{\lceil}}
\DeclareUnicodeCharacter{2309}{\ensuremath{\rceil}}
\DeclareUnicodeCharacter{230A}{\ensuremath{\lfloor}}
\DeclareUnicodeCharacter{230B}{\ensuremath{\rfloor}}
\DeclareUnicodeCharacter{25A1}{\ensuremath{\square}} % □
\DeclareUnicodeCharacter{25AF}{\mathop{\talloblong}} % ▯
\DeclareUnicodeCharacter{25C7}{\diamond} % ◇
\DeclareUnicodeCharacter{2605}{\ensuremath{\star}}   % ★
\DeclareUnicodeCharacter{2713}{\ensuremath{\checkmark}} % ✓
\DeclareUnicodeCharacter{27C2}{\ensuremath{^\bot}} % PERPENDICULAR ⟂
\DeclareUnicodeCharacter{27E6}{\ensuremath{\llbracket}} % ⟦
\DeclareUnicodeCharacter{27E7}{\ensuremath{\rrbracket}} % ⟧
\DeclareUnicodeCharacter{27E8}{\ensuremath{\langle}} % ⟨
\DeclareUnicodeCharacter{27E9}{\ensuremath{\rangle}} % ⟩
\DeclareUnicodeCharacter{27F6}{{\longrightarrow}} % ⟶
\DeclareUnicodeCharacter{27F7}{{\longleftrightarrow}} % ⟷
\DeclareUnicodeCharacter{2A04}{\mathop{\dot{\cup}}} % ⨄
\DeclareUnicodeCharacter{2AFE}{\mathop{\talloblong}} % ⫾

% \DeclareUnicodeCharacter{8499}{\mathcal{M}} 
% \DeclareUnicodeCharacter{8718}{\ensuremath{\blacksquare}}
% \DeclareUnicodeCharacter{8797}{\mathrel{\mathop:}=}
% \DeclareUnicodeCharacter{9657}{\ensuremath{\triangleright}}
% \DeclareUnicodeCharacter{9667}{\triangleright{}}
% \DeclareUnicodeCharacter{9669}{\ensuremath{\triangleleft}}


\title{A parametric Type theory}

\author{}
\date{\today}

\begin{document}
\maketitle


\begin{definition}[Syntax of terms and contexts]
  \begin{align*}
    t,u,A,B & \coloneqq x \mid U \mid λx:A. t      \mid t u \mid (x:A) → B \\
            & \mid \CP i t u  \mid \ip i A B  \mid \fp i t u \\
            & \mid A \op i u \mid t \param i  \\
    \Gamma,\Delta & \coloneqq () \mid \Gamma,x:A
  \end{align*}
\end{definition}

We consider a total order $\preccurlyeq$ on colours (for uniqueness of
normal forms).
We use $I,J,…$ for finite {\em sets} of colours (order does not matter).

\begin{definition}[Typing relation]
Contexts:
  \begin{mathpar}
    \inferrule[Empty]{~}{() ⊢ }

    \inferrule[NewVar]{Γ⊢ \\ Γ ⊢ A : U }{ Γ,x:A ⊢ }

  \end{mathpar}
 
Types:
 \begin{mathpar}
    \inferrule[Universe]{Γ ⊢}{Γ ⊢_I U : U}

    \inferrule[Pi]{Γ ⊢_I A : U \\ Γ,x:A ⊢_I B : U}{Γ ⊢_I (x:A) → B : U}

    \inferrule[Out]{Γ ⊢_{I,i} T : U \\ Γ ⊢_I a : \proj i T}{Γ ⊢_I T \op i  a : U}

    \inferrule[In-Pred]{Γ ⊢_{I} A : U \\ Γ ⊢_{I} P : A → U}{Γ ⊢_{I,i} \ip i A P : U}

 \end{mathpar}

Terms:
  \begin{mathpar}
    \inferrule[Conv]{Γ ⊢_I t:A \\ A = B}{Γ ⊢_I t : B}

    \inferrule[Var]{Γ ⊢ \\ x : A ∈ Γ}{Γ ⊢_I x : A}


    \\\inferrule[Lam]{Γ,x:A ⊢_I B}{Γ ⊢_I λx:A.b : (x:A) → B}
 
    \inferrule[App]{Γ ⊢_I t : (x:A) → B[x] \\ Γ ⊢_I u : A}{Γ ⊢_I t u: B[u]}

    \\\inferrule[In-Abs]{Γ ⊢_I a : \proj i T \\Γ ⊢_I p : T \op i a}{Γ ⊢_{I,i} \CP i a p : T}

    \inferrule[In-Fun]
        {Γ ⊢_I f : \proj i {(x:A) → P x}\\\\
        Γ ⊢_I g : (x:\proj i A) → (x':A \op i x) → P \CP i x {x'} \op i f x}{Γ ⊢_{I,i} \fp i f g : (x:A) → P x}

    \inferrule[Color-elim]{Γ ⊢_{I,i} a : T}{Γ ⊢_I a \param i : T \op i {\proj i a}}


  \end{mathpar}
\end{definition}

Notes:

\begin{itemize}
\item We collapse the universe hierarchy for conciseness, but
  supporting it is orthogonal to the specific features of this system.
\item We cannot have colored types in the environment.
\item If one wishes to define a function depending on a colored type
  one must therefore use the {\sc In-Fun} rule.
\item For example, the function $f = λx:A.b(x)$ is equivalent to $\fp
  i {proj i f} {λ(x:\proj i A). λ(x':A \op i x). (b \CP i x
    {x'}) \param i}$.
\item TODO: This means that the above is not strictly correct, we
  either need to define the Pi type as having an explicit function as
  its second argument, or have a special former corresponding to {\sc
    In-Fun}.
\end{itemize}

\begin{definition}[Projection]~
  $\proj i \cdot$ is defined by induction on the syntax.
\begin{align*}
    \proj i {\fp i f g} & = f \\
  \proj i {(a,_i p)} &= a \\
  \proj i {\ip i A P} &= A
\end{align*}
+ congruences, such as
\begin{align*}
  \proj i {T \op j a} &= {\proj i T \op j \proj i a}
\end{align*}
\end{definition}


\begin{definition}[Normal forms and neutral terms]~
  \begin{align*}
    \mathsf{Nf} ∋ u,v,A,B & \coloneqq
      U \mid λx:A. t \mid (x:A) → B \\
      & \mid \CP i u v \mid \fp i u v \\
      & \mid {(\ip {i₀} A B)} \op {i₁} {u_1 \cdots} \op {i_n} {u_n} &\quad \text{($i₀ \prec i₁ \prec \ldots \prec i_n$)} \\
      & \mid s \param {i₀} \cdots \param {i_{n-1}}                  &\quad \text{($i₀ \prec   < \ldots \prec i_{n-1}$)}
    \\
    \mathsf{Ne} ∋ s & \coloneqq x \mid s \, u
  \end{align*}
\end{definition}

\begin{definition}[Untyped reduction]~

β:
\begin{align*}
  (λx:A. u[x]) t &→ u[t]
\end{align*}

Swaps:
\begin{align*}
  a \param j \param i &→ a \param i \param j             &\text{($i \prec j$)} \\
  T \op j \CP i a c \op i b &→ T \op i \CP j a b \op j c &\text{($i \prec j$)}
\end{align*}
Note: the above rule shows why ${} \param i {}$ is different from $\op i {}$.

Pair-like things:
\begin{align*}
  {\fp i f g} \, a      &→ (f\,{\proj i a} ,_i g\,{\proj i a}\,{(a \param i)}) \\
  {(a,_i p)} \param i   &→ p \\
  {(\ip i A P)} \op i a &→ P\,a
\end{align*}

Note: the param-witness construction is bootstrapped by the last rule.

\end{definition}

Confluence implies other equalities, such as:
\begin{align*}
  {(a,_j p)} \param i &→ (a \param i ,_j p \param i)\\
  \ip j {(A \op i {\proj i a})} {(λy. P \, \CP i {\proj i a} y \op i {(a \param j)})} &→ {(\ip j A P)} \op i a &\text{($i \prec j$)}
\end{align*}

+ congruences.

\begin{definition}[Conversion]
  Let $↔^★$ be the reflexive symmetric transitive closure of $→$.
  \begin{mathpar}
    \inferrule{t ↔^★ u} {t = u}
    \and
    \inferrule{t x = u} {t = λ x:A.u}
    \and
    \inferrule{\proj i t = a \\ t \param i = p} {t = \CP i a p }
    \and
    \inferrule{\proj i t = f \\ (t \CP i x y) \param i = g x y} {t = \fp i f g }
    \and
    \inferrule{\proj i T = A \\ T \op i x = P x} {T = \CSig i A P }
  \end{mathpar}
\end{definition}




\section{Iterating Parametricity}
For any type $A$ (which may not depend on any color), we have
\begin{align*}
p &: (x:A) → A \op i x\\
p x &= x\param i\\
q &: (x:A) → (A \op i x) \op j x \param  i\\
  &: (x:A) → (A \op j \CP i x {x \param j}) \op j x \param  i\\
q x &= x\param i\param j
\end{align*}

If $A$ were to depend on a color, the types would not be well-formed:
types put in an environment must have no free color ({\sc New-Var}).

\section{Isomorphisms}


\begin{theorem}
$U \op i A$ is definitonally isomorphic to $A → U$.
\end{theorem}
\begin{proof}~
  \begin{enumerate}
  \item Assume $Q : U \op i A$. Then $P : A → U$ if $P x = \CP i A Q \op i x$.
  \item Assume $P : A → U$. Then $Q : U \op i A$ if $Q = (\CSig i A P) \param i$.
  \item Check 1: $\CP i A {(\CSig i A P) \param i} \op i x = (\CSig i A P) \op i x = P x$
  \item Check 2: $(\CSig i A {λx. \CP i A Q \op i x}).i = Q$ iff $\CSig i A {λx. \CP i A Q \op i x} = \CP i A Q$. We use conversion for $\Sigma$. The first components are obviously equal. For the second components we are left with $\CP i A Q \op i x = \CP i A Q \op i x$, which is obvious.
  \end{enumerate}
\end{proof}

\begin{theorem}
$((x:A) → B[x]) \op i f$ is definitonally isomorphic to $(x:A) → (x' : A \op i x) → B[\CP i x {x'}] \op i (f x)$.
\end{theorem}
\begin{proof}~
  \item Assume $q : ((x:A) → B[x]) \op i f$. Then $p : (x:A) → (x' : A \op i x) → B[\CP i x {x'}] \op i (f x)$ if $p x x' = (\CP i f q \CP i x {x'}) \param i$.
  \item Assume $p : (x:A) → (x' : A \op i x) → B[\CP i x {x'}] \op i (f x)$ Then $q : ((x:A) → B[x]) \op i f$ if $q = \fp i f p \param i$.
  \item Check 1: $(\CP i f {\fp i f p \param i} \CP i x x') \param i = ({\fp i f p} \CP i x {x'}) \param i = \CP i {f x} {p x x'} \param i = p x x' $
  \item Check 2: $\fp i f {λx x'. (\CP i f q \CP i x {x'}) \param i} \param i$ iff $\fp i f {λx x'. (\CP i f q \CP i x {x'}) \param i} = \CP i f q$, which is true by conversion of function pairing.
\end{proof}
\section{Link with "Old" parametricity (2)}
We define a (unary) relational interpretation of types $⟦A⟧$. This
interpretation coincides with the usual relational interpretation of types.
We show that, if $⟦·⟧$ and $∋_i$ are equivalent for free type variables,
then they are equivalent for every type in normal form.

\begin{definition}
We define $⟦A⟧$ by structural induction on $A$.
  \begin{align*}
    ⟦A → B⟧ f & = (x:\proj i A) → ⟦A⟧ x → ⟦B⟧ (f x)\\
    ⟦U⟧ A & = A → U\\
    ⟦A \op j a ⟧ b &= ⟦A⟧ \CP j {\proj i a} b \op j a \param i \\
    ⟦\ip i A P⟧ x & = P x\\
    ⟦x⟧ a & = x \op i a
  \end{align*}
\end{definition}

\providecommand\TO{\overrightarrow}
\providecommand\FROM{\overleftarrow}

\begin{theorem}
For every type $A$, $A \op i x$ is provable iff. $⟦A⟧ x$ is provable.
\end{theorem}
Let us call $\TO A$ the conversion from $A \op i x$ to $⟦A⟧ x$; and $\FROM A$ the other direction.
\begin{proof}
  By structural induction on $A$. Key cases:
  \begin{itemize}
  \item Pi.
    Assume:
    \begin{itemize}
    \item $f : A → B$
    \item $p : (A → B) \op i f$
    \end{itemize}
    We prove $q : (x:\proj i A) → ⟦A⟧ x → ⟦B⟧ (f x)$ with $q x x' = \TO {B[\CP i x {\FROM A x'}]} ((\CP i f
    p \CP i x {\FROM A x'}) \param i)$.

    Right to left: $p = (\fp i f {λx. λx'. \FROM {B[\CP i x {x'}]} (q x (\TO A x'))}) \param i $

    (No need to carry around an environment; the proof given by the
    user ($x'$) is substituted in terms --- the substitutions are
    well-typed, given the {\sc In-Abs} rule.)

  \item Universe.

    Left to right. Assume:
    \begin{itemize}
    \item $A : U$
    \item $P : U \op i A $
    \end{itemize}
    We prove $Q : A → U$ with $Q x = \CP i A P \op i x$.
 

    Right to left: $P = (\ip i A Q) \param i $

  \item Sigma.
    By def.
  \item Param.

    Left to right. Assume:
    \begin{itemize}
    \item $a : A$
    \item $b : A\op j a$
    \item $p : A \op j a \op i b $
    \end{itemize}
    We prove $q' : A \op i \CP j {\proj j a} b \op j a \param i  $ with $\CP i a {\CP j b p} \param j \param i$.
    The result is given by $\TO A q'$.

    Right to left: same principle.
  \end{itemize}
  \item Variable.
    By def.
\end{proof}

\subsection{Example: naturals}
Let $N = ∀X. X → (X → X) → X$.
Proving (unary) parametricity for $N$ means that, assuming
\begin{itemize}
\item $f : N$
\item $A : U$
\item $P : A → U$
\item $z : A$
\item $z' : P z$
\item $s : A → A$
\item $s' : (x:A) → P x → P (s x)$
\end{itemize}
we can prove $P (f A z s)$.

Indeed, a proof term is the following:

\[
(f (\ip i A P) \CP i z {z'} \fp i s {s'}) \param i
\]

\end{document}
