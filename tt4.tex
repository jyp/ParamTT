\documentclass[english]{PaperTools/latex/lipics}
\graphicspath{{PaperTools/latex/}}
\usepackage{natbib}
\usepackage[utf8]{inputenc}
\usepackage{stmaryrd}
\usepackage{amssymb,amstext,amsmath}
\usepackage{mathtools}
\usepackage{mathpartir}
\usepackage{fancyref}

\usepackage{tikz}
\usetikzlibrary{arrows}

\newcommand\CC[4]{(#2,_{#1} #3)}
\newcommand\CP[3]{(#2,_{#1} #3)}
\newcommand\CSig[1]{\Sigma^{#1}}
\newcommand\CTimes[3]{#2 ×_{#1} #3}
\newcommand\SW[2]{\mathsf{SW}^{#1}_{#2}}
\newcommand\sw[2]{\mathsf{sw}^{#1}_{#2}}
\newcommand\dom{\mathsf{dom}}
\newcommand\param[1]{\!\cdot\!#1}
\newcommand\pvar[2]{{#1}^{(#2)}}
\newcommand\op[1]{∋_{#1}}
\newcommand\ip[3]{Σ^{#1} {#2}\,{#3}}
\newcommand\fp[3]{⟨#2 ,_{#1} #3⟩}
\newcommand\mor[2]{{#1}\,{#2}}
\newcommand\proj[2]{\mor{#2}{(#1\,0)}}
\newcommand\projp[2]{\proj{#1}{(#2)}}

\newcommand\comment[1]{}
\DeclareUnicodeCharacter{00A0}{~} %   NO-BREAK SPACE
\DeclareUnicodeCharacter{00A7}{\S} % §
\DeclareUnicodeCharacter{00AC}{\ensuremath{\neg}} % ¬
\DeclareUnicodeCharacter{00B0}{^{\circ}} % °
\DeclareUnicodeCharacter{00B1}{^1} 
\DeclareUnicodeCharacter{00B2}{^2} % ²
\DeclareUnicodeCharacter{00B7}{\ensuremath{\cdot}} % ·
\DeclareUnicodeCharacter{00B9}{\textsuperscript{l}} % ¹
\DeclareUnicodeCharacter{00D7}{\ensuremath{\times}} % × 
\DeclareUnicodeCharacter{00F7}{\ensuremath{\div}} % ÷
\DeclareUnicodeCharacter{02E1}{\ensuremath{{^l}}} % ˡ
\DeclareUnicodeCharacter{02B3}{\ensuremath{{^r}}} % ʳ
\DeclareUnicodeCharacter{0393}{\ensuremath{\Gamma}} % Γ
\DeclareUnicodeCharacter{0394}{\ensuremath{\Delta}} % Δ
\DeclareUnicodeCharacter{0397}{\ensuremath{\textrm{H}}} % Η
\DeclareUnicodeCharacter{0398}{\ensuremath{\Theta}} % Θ
\DeclareUnicodeCharacter{039B}{\ensuremath{\Lambda}} % Λ
\DeclareUnicodeCharacter{039E}{\ensuremath{\Xi}} % Ξ
\DeclareUnicodeCharacter{03A3}{\ensuremath{\Sigma}} % Σ
\DeclareUnicodeCharacter{03A6}{\ensuremath{\Phi}} % Φ
\DeclareUnicodeCharacter{03A8}{\ensuremath{\Psi}} % Ψ
\DeclareUnicodeCharacter{03A9}{\ensuremath{\Omega}} % Ω
\DeclareUnicodeCharacter{03B1}{\ensuremath{\mathnormal{\alpha}}} % α
\DeclareUnicodeCharacter{03B2}{\ensuremath{\beta}} % β
\DeclareUnicodeCharacter{03B3}{\ensuremath{\mathnormal{\gamma}}} % γ
\DeclareUnicodeCharacter{03B4}{\ensuremath{\mathnormal{\delta}}} % δ
\DeclareUnicodeCharacter{03B5}{\ensuremath{\mathnormal{\varepsilon}}} % ε
\DeclareUnicodeCharacter{03B6}{\ensuremath{\mathnormal{\zeta}}} % ζ
\DeclareUnicodeCharacter{03B7}{\ensuremath{\mathnormal{\eta}}} % η
\DeclareUnicodeCharacter{03B8}{\ensuremath{\mathnormal{\theta}}} % θ
\DeclareUnicodeCharacter{03B9}{\ensuremath{\mathnormal{\iota}}} % ι
\DeclareUnicodeCharacter{03BA}{\ensuremath{\mathnormal{\kappa}}} % κ
\DeclareUnicodeCharacter{03BB}{\ensuremath{\mathnormal{\lambda}}} % λ
\DeclareUnicodeCharacter{03BC}{\ensuremath{\mathnormal{\mu}}} % μ
\DeclareUnicodeCharacter{03BD}{\ensuremath{\mathnormal{\mu}}} % ν
\DeclareUnicodeCharacter{03BE}{\ensuremath{\mathnormal{\xi}}} % ξ
\DeclareUnicodeCharacter{03C0}{\ensuremath{\mathnormal{\pi}}} % π
\DeclareUnicodeCharacter{03C1}{\ensuremath{\mathnormal{\rho}}} % ρ
\DeclareUnicodeCharacter{03C3}{\ensuremath{\mathnormal{\sigma}}} % σ
\DeclareUnicodeCharacter{03C4}{\ensuremath{\mathnormal{\tau}}} % τ
\DeclareUnicodeCharacter{03C6}{\ensuremath{\mathnormal{\varphi}}} % φ
\DeclareUnicodeCharacter{03D5}{\ensuremath{\mathnormal{\phi}}} % ϕ
\DeclareUnicodeCharacter{03C7}{\ensuremath{\mathnormal{\chi}}} % χ
\DeclareUnicodeCharacter{03C8}{\ensuremath{\mathnormal{\psi}}} % ψ
\DeclareUnicodeCharacter{03C9}{\ensuremath{\mathnormal{\omega}}} % ω 
\DeclareUnicodeCharacter{03F5}{\ensuremath{\mathnormal{\epsilon}}} % ϵ
\DeclareUnicodeCharacter{1D62}{_i} % ᵢ
\DeclareUnicodeCharacter{10627}{\ensuremath{\lbana}} 
\DeclareUnicodeCharacter{10628}{\ensuremath{\rbana}} 
\DeclareUnicodeCharacter{2026}{\ensuremath{\ldots}}
\DeclareUnicodeCharacter{202F}{{\,}}
\DeclareUnicodeCharacter{2080}{\ensuremath{_0}} % ₀
\DeclareUnicodeCharacter{2081}{\ensuremath{_1}}
\DeclareUnicodeCharacter{2082}{\ensuremath{_2}}
\DeclareUnicodeCharacter{2083}{\ensuremath{_3}}
\DeclareUnicodeCharacter{2084}{\ensuremath{_4}}
\DeclareUnicodeCharacter{2085}{\ensuremath{_5}}
\DeclareUnicodeCharacter{2086}{\ensuremath{_6}}
\DeclareUnicodeCharacter{2087}{\ensuremath{_7}}
\DeclareUnicodeCharacter{2088}{\ensuremath{_8}}
\DeclareUnicodeCharacter{2089}{\ensuremath{_9}}
\DeclareUnicodeCharacter{2115}{\mathbb{N}}
\DeclareUnicodeCharacter{214B}{\ensuremath{\parr}}
\DeclareUnicodeCharacter{2190}{\ensuremath{\leftarrow}} % ← 
\DeclareUnicodeCharacter{2191}{\ensuremath{\uparrow}} % ↑
\DeclareUnicodeCharacter{2192}{\ensuremath{\rightarrow}} % →
\DeclareUnicodeCharacter{2194}{\ensuremath{\leftrightarrow}} % ↔
\DeclareUnicodeCharacter{2196}{\nwarrow} % ↖
\DeclareUnicodeCharacter{2197}{\nearrow} % ↗
\DeclareUnicodeCharacter{219D}{\ensuremath{\leadsto}} % ↝
\DeclareUnicodeCharacter{21A6}{\ensuremath{\mapsto}} % ↦ 
\DeclareUnicodeCharacter{21C6}{\ensuremath{\leftrightarrows}} % ⇆
\DeclareUnicodeCharacter{21D0}{\ensuremath{\Leftarrow}} % ⇐
\DeclareUnicodeCharacter{21D2}{\ensuremath{\Rightarrow}} % ⇒ 
\DeclareUnicodeCharacter{21D4}{\ensuremath{\Leftrightarrow}} % ⇔
\DeclareUnicodeCharacter{2200}{\ensuremath{\forall}} % ∀
\DeclareUnicodeCharacter{2203}{\ensuremath{\exists}} % ∃
\DeclareUnicodeCharacter{2205}{\ensuremath{\varnothing}} % ∅
\DeclareUnicodeCharacter{2208}{\ensuremath{\in}} % ∈
\DeclareUnicodeCharacter{2209}{\ensuremath{\not\in}} % ∉
\DeclareUnicodeCharacter{220B}{\ensuremath{\ni}}
\DeclareUnicodeCharacter{220E}{\ensuremath{\qed}} % ∎ % Alternatively use \blacksquare
\DeclareUnicodeCharacter{2211}{\sum}% ∑
\DeclareUnicodeCharacter{2215}{\mathbb{N}} % ℕ
\DeclareUnicodeCharacter{2217}{\ensuremath{\ast}} % ∗
\DeclareUnicodeCharacter{2218}{\ensuremath{\circ}} % ∘
\DeclareUnicodeCharacter{2219}{\ensuremath{\bullet}} % ∙ 
\DeclareUnicodeCharacter{221E}{\ensuremath{\infty}} % ∞
\DeclareUnicodeCharacter{2223}{\ensuremath{\mid}} % ∣
\DeclareUnicodeCharacter{2227}{\wedge}% ∧
\DeclareUnicodeCharacter{2228}{\vee}% ∨
\DeclareUnicodeCharacter{2229}{\ensuremath{\cap}} % ∩
\DeclareUnicodeCharacter{222A}{\ensuremath{\cup}} % ∪
\DeclareUnicodeCharacter{2237}{::} % ∷
\DeclareUnicodeCharacter{223C}{\ensuremath{\sim}} % ∼
\DeclareUnicodeCharacter{2243}{\ensuremath{\simeq}} % ≃
\DeclareUnicodeCharacter{2245}{\ensuremath{\cong}} % ≅ 
\DeclareUnicodeCharacter{2248}{\ensuremath{\approx}} % ≈
\DeclareUnicodeCharacter{225C}{\ensuremath{\stackrel{\scriptscriptstyle {\triangle}}{=}}} % ≜
\DeclareUnicodeCharacter{225F}{\ensuremath{\stackrel{\scriptscriptstyle ?}{=}}} % ≟
\DeclareUnicodeCharacter{2260}{\neq}% ≠
\DeclareUnicodeCharacter{2261}{\equiv}% ≡
\DeclareUnicodeCharacter{2264}{\ensuremath{\le}} % ≤
\DeclareUnicodeCharacter{2265}{\ensuremath{\ge}} % ≥
\DeclareUnicodeCharacter{2282}{\ensuremath{\subset}} % ⊂
\DeclareUnicodeCharacter{2283}{\ensuremath{\supset}} % ⊃ 
\DeclareUnicodeCharacter{2286}{\ensuremath{\subseteq}} % ⊆ 
\DeclareUnicodeCharacter{2287}{\ensuremath{\supseteq}} % ⊇
\DeclareUnicodeCharacter{2293}{\ensuremath{\sqcup}} % ⊓
\DeclareUnicodeCharacter{2293}{\sqcap} % ⊓
\DeclareUnicodeCharacter{2294}{\sqcup} % ⊔
\DeclareUnicodeCharacter{2295}{\ensuremath{\oplus}} % ⊕
\DeclareUnicodeCharacter{2297}{\ensuremath{\otimes}} % ⊗
\DeclareUnicodeCharacter{22A2}{\ensuremath{\vdash}}
\DeclareUnicodeCharacter{22A4}{\ensuremath{\top}} % ⊤
\DeclareUnicodeCharacter{22A5}{\ensuremath{\bot}} % ⊥
\DeclareUnicodeCharacter{22A7}{\models} % ⊧ 
\DeclareUnicodeCharacter{22A8}{\models} % ⊨
\DeclareUnicodeCharacter{22A9}{\Vdash} % ⊩
\DeclareUnicodeCharacter{22B8}{\ensuremath{\multimap}} % ⊸
\DeclareUnicodeCharacter{22C4}{\diamond} % ⋄
\DeclareUnicodeCharacter{22C6}{\ensuremath{\star}}
\DeclareUnicodeCharacter{22EE}{\ensuremath{\vdots}} % ⋮
\DeclareUnicodeCharacter{22EF}{\ensuremath{\cdots}} % ⋯
\DeclareUnicodeCharacter{2308}{\ensuremath{\lceil}}
\DeclareUnicodeCharacter{2309}{\ensuremath{\rceil}}
\DeclareUnicodeCharacter{230A}{\ensuremath{\lfloor}}
\DeclareUnicodeCharacter{230B}{\ensuremath{\rfloor}}
\DeclareUnicodeCharacter{25A1}{\ensuremath{\square}} % □
\DeclareUnicodeCharacter{25AF}{\mathop{\talloblong}} % ▯
\DeclareUnicodeCharacter{25C7}{\diamond} % ◇
\DeclareUnicodeCharacter{2605}{\ensuremath{\star}}   % ★
\DeclareUnicodeCharacter{2713}{\ensuremath{\checkmark}} % ✓
\DeclareUnicodeCharacter{27C2}{\ensuremath{^\bot}} % PERPENDICULAR ⟂
\DeclareUnicodeCharacter{27E6}{\ensuremath{\llbracket}} % ⟦
\DeclareUnicodeCharacter{27E7}{\ensuremath{\rrbracket}} % ⟧
\DeclareUnicodeCharacter{27E8}{\ensuremath{\langle}} % ⟨
\DeclareUnicodeCharacter{27E9}{\ensuremath{\rangle}} % ⟩
\DeclareUnicodeCharacter{27F6}{{\longrightarrow}} % ⟶
\DeclareUnicodeCharacter{27F7}{{\longleftrightarrow}} % ⟷
\DeclareUnicodeCharacter{2A04}{\mathop{\dot{\cup}}} % ⨄
\DeclareUnicodeCharacter{2AFE}{\mathop{\talloblong}} % ⫾

% \DeclareUnicodeCharacter{8499}{\mathcal{M}} 
% \DeclareUnicodeCharacter{8718}{\ensuremath{\blacksquare}}
% \DeclareUnicodeCharacter{8797}{\mathrel{\mathop:}=}
% \DeclareUnicodeCharacter{9657}{\ensuremath{\triangleright}}
% \DeclareUnicodeCharacter{9667}{\triangleright{}}
% \DeclareUnicodeCharacter{9669}{\ensuremath{\triangleleft}}


\def\pI{\ensuremath{\mathbf{pI}}}
\def\fresh#1{\mathsf{fresh}(#1)}
\def\Hom#1#2{\mathbf{Hom}(#1,#2)}
\def\ie{\textit{i.e.}}
\def\app#1#2{\mathsf{app}(#1,#2)}
\def\El#1{\mathrm{El}(#1)}

\title{A presheaf model of parametric type-theory}

\author{}
\date{\today}

\begin{document}
\maketitle

\begin{abstract}
  We propose a new intentional type-theory with internalized
  parametricity. Compared to previous similar proposals, this version
  comes with a denotational semantics, which is a refinement of the
  standard presheaf semantics of dependent type theory. This presheaf
  semantics is a refinement of the one used to interpret nominal sets
  with restriction.
  This calculus also introduces fewer syntactic elements than the
  previous attempts, making it a better candidate for the core of a
  proof assistant with internalized parametricity.
\end{abstract}

\section{Syntax}
In the following section we define the syntax and typing-rules of our
parametric type-theory, as well as the equality judgement.

The metasyntactic variables $i,j,\ldots$ range over colors, while
$I,J,…$ range over finite sets of colours.

\begin{definition}[Syntax of terms and contexts]
  \begin{align*}
    t,u,A,B & \coloneqq x \mid U \mid λx:A. t      \mid t u \mid (x:A) → B \\
            & \mid \CP i t u  \mid \ip i A B  \mid \fp i t u \\
            & \mid A \op i u \mid t \param i  \\
    \Gamma,\Delta & \coloneqq () \mid \Gamma,x:A
  \end{align*}
\end{definition}

The crucial additions to the syntax are the following:
\begin{itemize}

\item the type $A \op i u$, which expresses that $u$ satisfies the
  parametricity predicate associated with the type $A$ on color $i$.
\item the term $a \param i$, which yields proofs of parametricity for
  the type $A$.
\item the forms $\CP i t u$, $\ip i A B$ and $\fp i t u$ allow to locally override associate spefic parametricity proofs.
\end{itemize}


\begin{definition}[Typing judgements — à la Tarski]
Contexts:
  \begin{mathpar}
    \inferrule[Empty]{~}{() ⊢_I }

    \inferrule[NewVar]{Γ⊢_I \\ Γ ⊢_I A }{ Γ,x:A ⊢_I }

  \end{mathpar}
 
Types:
 \begin{mathpar}
    \inferrule{Γ ⊢_I \text{$A$ small} }{Γ ⊢_I |A| : U}

    \inferrule{Γ ⊢_I A : U}{Γ ⊢_I \El{A}}

    \El{|A|} = A

    |\El{A}| = A
    \\
    \inferrule[Pi]{Γ ⊢_I A \\ Γ,x:A ⊢_I B}{Γ ⊢_I (x:A) → B}

    \inferrule[Out]{Γ ⊢_{I,i} T \\ Γ ⊢_I a : \proj i T}{Γ ⊢_I T \op i  a}

    \inferrule[In-Pred]{Γ ⊢_{I} A \\ Γ,x:A ⊢_{I} P}{Γ ⊢_{I,i} \ip i {(x:A)} P}
 \end{mathpar}

Terms:
  \begin{mathpar}
    \inferrule[Conv]{Γ ⊢_I t:A \\ A = B}{Γ ⊢_I t : B}

    \inferrule[Var]{Γ ⊢_I \\ x : A ∈ Γ}{Γ ⊢_I x : A}


    \\\inferrule[Lam]{Γ,x:A ⊢_I B}{Γ ⊢_I λx:A.b : (x:A) → B}
 
    \inferrule[App]{Γ ⊢_I t : (x:A) → B[x] \\ Γ ⊢_I u : A}{Γ ⊢_I t u: B[u]}

    \\\inferrule[In-Abs]{Γ ⊢_I a : \proj i T \\Γ ⊢_I p : T \op i a}{Γ ⊢_{I,i} \CP i a p : T}

    \inferrule[In-Fun]
        {Γ ⊢_I f : \projp i {(x:A) → P x}\\\\
        Γ ⊢_I g : (x:\proj i A) → (x':A \op i x) → P \CP i x {x'} \op i f x}{Γ ⊢_{I,i} \fp i f g : (x:A) → P x}

    \inferrule[Color-elim]{Γ ⊢_{I,i} a : T}{Γ ⊢_I a \param i : T \op i {\proj i a}}


  \end{mathpar}
\end{definition}

Notes:

\begin{itemize}
  \item The composition of the coding function with $\El{·}$ is the
    identity: $\El{|A|} = A$ and $|\El{A}| = A$.
\item If one wishes to define a function depending usefully on a colored type
  need to use the {\sc In-Fun} rule.
\end{itemize}

\begin{definition}[Morphism application]~
  We consider partial injections $f : I → J$ and for any $I$-term $a$,
  we define the $J$-term $af$ by structural induction on $a$.
  We note $(ι\,0) : I,ι → I$ the partial identity.
\begin{align*}
  x f & = x \\
  U f & = U \\
  (λ(x:A).t) f &= λ(x:Af).tf \\
  (t\,u) f &= (tf) \, (uf) \\
  ((x:A)→B) f &= (x:Af)→(Bf) \\
  \CP {ι} a p f &= \CP {κ} {ag} {pg}
    & \text{if $fι = κ$, $g : I→J_κ$ such that $(ι\,0)g = f(κ\,0) : I,ι → J_κ$} \\
                &= ag
    & \text{if $∃g : I→J$ such that $f = (ι\,0)g : I,ι → J$} \\
  (\ip {ι} A P) f &= \ip {κ} {(Ag)} {(Pg)}
    & \text{if $fι = κ$, $g : I→J_κ$ such that $(ι\,0)g = f(κ\,0) : I,ι → J_κ$} \\
                &= Ag
    & \text{if $∃g : I→J$ such that $f = (ι\,0)g : I,ι → J$} \\
  \fp {ι} u v f &= \fp {κ} {ug} {vg}
    & \text{if $fι = κ$, $g : I→J_κ$ such that $(ι\,0)g = f(κ\,0) : I,ι → J_κ$} \\
                &= ug
    & \text{if $∃g : I→J$ such that $f = (ι\,0)g : I,ι → J$} \\
  A \op {ι} a &= (Ag) \op {κ} (af)
    & \text{where $κ = \fresh J$ and $g:I,ι → J,κ$ satisfies $(ι\,0)f = g(κ\,0)$} \\
  (a · ι) f &= (ag) · κ
    & \text{where $κ = \fresh J$ and $g:I,ι → J,κ$ satisfies $(ι\,0)f = g(κ\,0)$}
\end{align*}
  (We remark that for any $f : I,ι → J,κ$ with $fι=κ$, there is a
  unique $g : I → J$ such that $(ι\,0)g = f(κ\,0) : I,ι → J$;
  and that for any $f : I,ι → J$ which is not defined on $ι$, there is
  a unique $g : I → J$ such that $(ι\,0)g = f$.)

  We have $a1 = a$ and $(af)g = a(fg)$ for any $g : J → K$.
  Furthermore, if $Γ ⊢_I a : A$ then $Γ f ⊢_J af : Af$ (proof by
  induction on the typing judgement).
\end{definition}


\comment{
\begin{definition}[Normal forms and neutral terms]~
  \begin{align*}
    \mathsf{Nf} ∋ u,v,A,B & \coloneqq
      U \mid λx:A. t \mid (x:A) → B \\
      & \mid \CP i u v \mid \fp i u v \\
      & \mid {(\ip {i₀} A B)} \op {i₁} {u_1 \cdots} \op {i_n} {u_n} &\quad \text{($i₀ \prec i₁ \prec \ldots \prec i_n$)} \\
      & \mid s \param {i₀} \cdots \param {i_{n-1}}                  &\quad \text{($i₀ \prec   < \ldots \prec i_{n-1}$)}
    \\
    \mathsf{Ne} ∋ s & \coloneqq x \mid s \, u
  \end{align*}
\end{definition}
}
\begin{definition}[Conversion]~
\label{def:conversion}

β:
\begin{align*}
  (λx:A. u[x]) t &= u[t]
\end{align*}

Pair-like things:
\begin{align*}
  {\fp i f g} \, a      &= (f\,{\proj i a} ,_i g\,{\proj i a}\,{(a \param i)}) \\
  {(a,_i p)} \param i   &= p \\
  {(\ip i A P)} \op i a &= P\,a
\end{align*}

Note: the param-witness construction is bootstrapped by the last rule.

  \begin{mathpar}
    \inferrule{t x = u} {t = λ x:A.u}
    \and
    \inferrule{\proj i t = a \\ t \param i = p} {t = \CP i a p }
    \and
    \inferrule{\proj i t = f \\ (t \CP i x y) \param i = g x y} {t = \fp i f g }
    \and
    \inferrule{\proj i T = A \\ T \op i x = P x} {T = \CSig i A P }
    \and
    \inferrule{~}{a = a}
    \and
    \inferrule{a = b}{b = a}
    \and
    \inferrule{a = b \\ b = c}{a = c}
  \end{mathpar}
  + congruences
\end{definition}

\paragraph{Consequences}
\begin{corollary}~
  \begin{itemize}
  \item $a = \CP i {\proj i a} {a \param i}$
  \item $T = \CSig i {\proj i T} {(λx. T \op i x)}$
  \item $f = \fp i {\proj i f} {λx x'. (f \CP i x {x'}) \param i}$
  \end{itemize}
\end{corollary}

\section{Iterating Parametricity}
For any type $A$ (which may not depend on any color), we have
\begin{align*}
p &: (x:A) → A \op i x\\
p x &= x\param i\\
q &: (x:A) → (A \op i x) \op j x \param  i\\
  &: (x:A) → (A \op j \CP i x {x \param j}) \op j x \param  i\\
q x &= x\param i\param j
\end{align*}

If $A$ were to depend on a color, the types would not be well-formed:
types put in an environment must have no free color ({\sc New-Var}).

\section{Isomorphisms}

We say that two types are definitionally isomorphic iff. they are
isomorphic and that the equalities use the conversion relation
(\ref{def:conversion}).

\begin{theorem}
$U \op i A$ is definitionally isomorphic to $A → U$.
\end{theorem}
\begin{proof}~
  \begin{enumerate}
  \item Assume $Q : U \op i A$. Then $P : A → U$ if $P x = \CP i A Q \op i x$.
  \item Assume $P : A → U$. Then $Q : U \op i A$ if $Q = (\CSig i A P) \param i$.
  \item Check 1: $\CP i A {(\CSig i A P) \param i} \op i x = (\CSig i A P) \op i x = P x$
  \item Check 2: $(\CSig i A {λx. \CP i A Q \op i x}).i = Q$ iff $\CSig i A {λx. \CP i A Q \op i x} = \CP i A Q$. We use conversion for $\Sigma$. The first components are obviously equal. For the second components we are left with $\CP i A Q \op i x = \CP i A Q \op i x$, which is obvious.
  \end{enumerate}
\end{proof}

\begin{theorem}
$((x:A) → B[x]) \op i f$ is definitionally isomorphic to $(x:A) → (x' : A \op i x) → B[\CP i x {x'}] \op i (f x)$.
\end{theorem}
\begin{proof}~
  \item Assume $q : ((x:A) → B[x]) \op i f$. Then $p : (x:A) → (x' : A \op i x) → B[\CP i x {x'}] \op i (f x)$ if $p x x' = (\CP i f q \CP i x {x'}) \param i$.
  \item Assume $p : (x:A) → (x' : A \op i x) → B[\CP i x {x'}] \op i (f x)$ Then $q : ((x:A) → B[x]) \op i f$ if $q = \fp i f p \param i$.
  \item Check 1: $(\CP i f {\fp i f p \param i} \CP i x x') \param i = ({\fp i f p} \CP i x {x'}) \param i = \CP i {f x} {p x x'} \param i = p x x' $
  \item Check 2: $\fp i f {λx x'. (\CP i f q \CP i x {x'}) \param i} \param i$ iff $\fp i f {λx x'. (\CP i f q \CP i x {x'}) \param i} = \CP i f q$, which is true by conversion of function pairing.
\end{proof}
\section{Link with "Old" parametricity (2)}
We define a (unary) relational interpretation of types $⟦A⟧$. This
interpretation coincides with the usual relational interpretation of types.
We show that, if $⟦·⟧$ and $∋_i$ are equivalent for free type variables,
then they are equivalent for every type in normal form.

\begin{definition}
We define $⟦A⟧$ by structural induction on $A$.
  \begin{align*}
    ⟦A → B⟧ f & = (x:\proj i A) → ⟦A⟧ x → ⟦B⟧ (f x)\\
    ⟦U⟧ A & = A → U\\
    ⟦A \op j a ⟧ b &= ⟦A⟧ \CP j {\proj i a} b \op j a \param i \\
    ⟦\ip i A P⟧ x & = P x\\
    ⟦x⟧ a & = x \op i a
  \end{align*}
\end{definition}

\providecommand\TO{\overrightarrow}
\providecommand\FROM{\overleftarrow}

\begin{theorem}
For every type $A$, $A \op i x$ is provable iff. $⟦A⟧ x$ is provable.
\end{theorem}
Let us call $\TO A$ the conversion from $A \op i x$ to $⟦A⟧ x$; and $\FROM A$ the other direction.
\begin{proof}
  By structural induction on $A$. Key cases:
  \begin{itemize}
  \item Pi.
    Assume:
    \begin{itemize}
    \item $f : A → B$
    \item $p : (A → B) \op i f$
    \end{itemize}
    We prove $q : (x:\proj i A) → ⟦A⟧ x → ⟦B⟧ (f x)$ with $q x x' = \TO {B[\CP i x {\FROM A x'}]} ((\CP i f
    p \CP i x {\FROM A x'}) \param i)$.

    Right to left: $p = (\fp i f {λx. λx'. \FROM {B[\CP i x {x'}]} (q x (\TO A x'))}) \param i $

    (No need to carry around an environment; the proof given by the
    user ($x'$) is substituted in terms --- the substitutions are
    well-typed, given the {\sc In-Abs} rule.)

  \item Universe.

    Left to right. Assume:
    \begin{itemize}
    \item $A : U$
    \item $P : U \op i A $
    \end{itemize}
    We prove $Q : A → U$ with $Q x = \CP i A P \op i x$.
 

    Right to left: $P = (\ip i A Q) \param i $

  \item Sigma.
    By def.
  \item Param.

    Left to right. Assume:
    \begin{itemize}
    \item $a : A$
    \item $b : A\op j a$
    \item $p : A \op j a \op i b $
    \end{itemize}
    We prove $q' : A \op i \CP j {\proj j a} b \op j a \param i  $ with $\CP i a {\CP j b p} \param j \param i$.
    The result is given by $\TO A q'$.

    Right to left: same principle.
  \end{itemize}
  \item Variable.
    By def.
\end{proof}

\subsection{Example: naturals}
Let $N = ∀X. X → (X → X) → X$.
Proving (unary) parametricity for $N$ means that, assuming
\begin{itemize}
\item $f : N$
\item $A : U$
\item $P : A → U$
\item $z : A$
\item $z' : P z$
\item $s : A → A$
\item $s' : (x:A) → P x → P (s x)$
\end{itemize}
we can prove $P (f A z s)$.

Indeed, a proof term is the following:

\[
(f (\ip i A P) \CP i z {z'} \fp i s {s'}) \param i
\]

\section{Presheaf model}

\begin{definition}
  Let \pI{} be the category of finite set of names $I,J,K,…$ and injective
  partial function \cite[ex.~9.7 p.~176]{PittsAM:nomsns}.
  (Any morphism $f : I → J$ has a unique decomposition into a projection map
  $α : I → I_α$, which is not defined anywhere, and an injection map $h : I_α ↣ J$.)
  For any object $I$ and $ι ∉ I$, we note $(ι \, 0) : I,ι → I$ the partial
  identity on $I$.

  \begin{tikzpicture}[node distance=4\baselineskip]
    \node              (I)  {$I$};
    \node[below of=I]  (Ia) {$I_α$};
    \node[right of=Ia] (J)  {$J$};

    \draw[->] (I) to node[left] {$α$} (Ia);
    \draw[->] (I) to node[above right] {$f$} (J);
    \draw[to reversed->] (Ia) to node[below] {$h$} (J);
  \end{tikzpicture}

  For any object $I$, let $\fresh{I}$ be a fresh name: $\fresh{I} ∉ I$.
\end{definition}

\begin{definition}
  We call $I$-element any tuple indexed by the subsets of $I$: $(u_J)_{J ⊆ I}$.
  An $I$-set is a set of $I$-elements.  For instance, the elements of a
  $\{ι,κ\}$-set are of the form $u = (u_∅,u_ι,u_κ,u_{ι,κ})$.
  %
  If $a,b$ are $I$-elements and $κ ∉ I$, we define the $(I,κ)$-element
  $(a ,_κ b)$ as $(a ,_κ b)_J ≔ a_J$ if $κ ∉ J$ and $(a ,_κ b)_{J,κ} ≔ b_J$.
  %
  Any $(I,ι)$-element can be written $u = (u_J)_{J ⊆ \{I,ι\}} = (u_J)_{J ⊆ I} ∪ (u_{J,ι})_{J ⊆ I}$;
  We can therefore define the $I$-elements $u (ι\,0) ≔ (u_J)_{J ⊆ I}$ and $u · ι ≔ (u_{J,ι})_{J ⊆ I}$.
  (Hence by definition $u = (u (ι\,0) ,_ι u · ι)$.)
\end{definition}

\bigskip
We take the the usual presheaf model of type theory on \pI{} with the
following restrictions:
\begin{enumerate}
  \item for any $I ∈ \pI$, $F(I)$ is a $I$-set, and
  \item for any projection map $α : I → I_α$, the
    map $u ↦ uα$, $F(I) → F(I_α)$ is the projection operation, \ie,
    $uα_J = u_J$ for any $J ⊆ I$.
\end{enumerate}

\bigskip
A context $Γ ⊢_I$ is given by a $J$-set $Γf$ for each map $f : I → J$.
Furthermore the map $ρ ↦ ρα$, $Γf → Γ(fα)$ is the projection operation,
and if $g : J → K$ then the map $ρ ↦ ρg$, $Γf → Γ(fg)$ is such that
$ρ1 = ρ$ and $(ρg)h = ρ(gh)$ for any $h : K → L$.

\medskip
A type $Γ ⊢_I A$ is given by a $J$-set $A(f,ρ)$ for each $f : I → J$ and
$ρ : Γf$.
Furthermore the map $u ↦ uα$, $A(f,ρ) → A(fα,ρα)$ is the projection operation,
and if $g : J → K$ then the map $u ↦ ug$, $A(f,ρ) → A(fg,ρg)$ is such that
$u1 = u$ and $(ug)h = u(gh)$ for any $h : K → L$.

\medskip
A term $Γ ⊢_I a : A$ is given by a family $a(f,ρ) : A(f,ρ)$ such that
$a(f,ρ)g = a(fg,ρg)$ for any $f : I → J$, $ρ : Γf$ and $g : J → K$.

\medskip
If $Γ ⊢_I A$ and $f : I → J$, we define $⟨ρ,u⟩ : (Γ.A)f$ to mean
$ρ ∈ Γf$ and $u ∈ A(f,ρ)$.


\bigskip
\begin{description}
  \item[\sc Universe.]
    Let $f : I → J$, $ρ : Γf$.  We interpret $A(f,ρ) : U(f,ρ)$ as the
    $J$-element $u = \left(\left(ω(αh)\right)_{h : J_α ↣ K}\right)_{J_α ⊆ J}$,
    where $ωg = A(fg,ρg)$.
    For $g : J → K$, we define the $K$-element $u g$ as
    $\left(\left(ω(gαh)\right)_{h : K_α ↣ L}\right)_{K_α ⊆ K}$.

    The restriction maps are defined by induction on $I$.
    For $I = ∅$ we have $U(∅) = (Af)_{f:∅ ↣ J}$ and we take the map
    $u ↦ uf$, $A(∅) → A(J)$.
    For $I,ι$ we have $U(I,ι) = \left(\left(A(αh)\right)_{h : I_α ↣ J}\right)_{I_α ⊆ I,ι}
    = \left(\left(A(α_1h)\right)_{h : I_α ↣ J}\right)_{I_α ⊆ I} ∪
      \left(\left(A(α_2h)\right)_{h : I_α,ι ↣ J}\right)_{I_α ⊆ I}$
    where $α_1 : I,ι → I_α$ and $α_2 : I,ι → I_α,ι$ are the projections
    maps.
    The map $u ↦ uf$, $A(I,ι) → A(J)$ is
    that from $A(I) → A(J)$ if $f$ is not defined on $ι$; otherwise
    we have $κ = fι$ and $g : I→J$ such that $(ι\,0)g = f(κ\,0)$, and we
    take $u ↦ ug$.

    Note that the $J$-element $u$ is isomorphic to the usual interpretation of
    $U(f,ρ)$, the family $ω$ indexed by $g : J \stackrel{α}{→} J_α \stackrel {h}{↣} K$.
    Indeed repartitioning $ω$ gives
    $\left(ωg\right)_{g : J → K} ≅ \left(\left(ω(gαh)\right)_{h : K_α ↣ L}\right)_{K_α ⊆ K}$.

    Consider the map $u ↦ uβ$, $U(f,ρ) → U(fβ,ρβ)$.  Let
    $K ⊆ J_β$, and $γ : J \stackrel{β}{→} J_β \stackrel{α}{→} K$ the projection map.
    We have that
    $u_K = \left(ω(γh)\right)_{h : K ↣ L}
         = \left(ω(βαh)\right)_{h : K ↣ L}
         = uβ_K$.

  \item[\sc Pi.]
    Let $f : I → J$, $ρ : Γf$.  A ($J$-)element $(Π A B)(f,ρ)$ is defined as
    $w = \left(\left(λ(αh)\right)_{h : J_α ↣ K}\right)_{J_α ⊆ J}$,
    where
    $$λ g = \prod_{u : A(fg,ρg)} B(fg,⟨ρg,u⟩) \quad\text{for $g : J → K$}$$
    such that
    $\app{λg} u h = \app{λ(gh)}{uh}$ for $g : J → K$ and $h : K → L$.
    We define the $K$-element $w g$ as
    $\left(\left(λ(gαh)\right)_{h : K_α ↣ L}\right)_{K_α ⊆ K}$ for any $g : J → K$.

    Note that the $J$-element $w$ is isomorphic to the usual interpretation
    of $(ΠAB)(f,ρ)$, a family $λ$ of functions indexed by
    $g : J \stackrel{α}{→} J_α \stackrel {h}{↣} K$.  Indeed repartitioning $λ$ gives
    $\left(λg\right)_{g : J → K} ≅ \left(\left(λ(αh)\right)_{h : J_α ↣ K}\right)_{J_α ⊆ J}$.

    Consider the map $w ↦ wβ$, $(Π A B)(f,ρ) → (Π A B)(fβ,ρβ)$.  Let
    $K ⊆ J_β$, and $γ : J \stackrel{β}{→} J_β \stackrel{α}{→} K$ the projection map.
    We have that
    $w_K = \left(λ(γh)\right)_{h : K ↣ L}
         = \left(λ(βαh)\right)_{h : K ↣ L}
         = wβ_K$.

%    Taking $ρ : Γ(ι\,0)$, we get by induction that
%    that the ($I$-)elements of $(Π A B)((ι\,0),ρ)$
%    are those of $(Π (A(ι\,0)) (B(ι\,0)))(1,ρ)$, \ie, of $(Π A B)(ι\,0)(1,ρ)$:
%    $$ \left(\left(\prod_{u : A((ι\,0)αh,ραh)} B((ι\,0)αh,⟨ραh,u⟩)\right)_{h : I_α ↣ K}\right)_{I_α ⊆ I}
%      = \left(\left(\prod_{u : A(ι\,0)(αh,ραh)} B(ι\,0)(αh,⟨ραh,u⟩)\right)_{h : I_α ↣ K}\right)_{I_α ⊆ I}
%    $$
%    If $g : J \stackrel{α}{→} J_α \stackrel {h}{↣} K$, we note $wg$ the
%    function $$\prod_{u : A(fαh,ραh)} B(fαh,⟨ραh,u⟩)$$.

  \item[\sc Out.]
    Let $f : I → J$, $ρ : Γf$.  We need to define the $J$-set $(P \op {ι} a)(f,ρ)$.
    Let $κ = \fresh J$ and $f' : I,ι → J,κ$ such that the following
    diagram commutes:

    \begin{tikzpicture}[node distance=4\baselineskip]
      \node              (Ii) {$I,ι$};
      \node[below of=Ii] (I)  {$I$};
      \node[right of=Ii] (Jk) {$J,κ$};
      \node[below of=Jk] (J)  {$J$};

      \draw[->] (Ii) to node[left]  {$(ι\,0)$} (I);
      \draw[->] (Ii) to node[above] {$f'$}     (Jk);
      \draw[->] (Jk) to node[right] {$(κ\,0)$} (J);
      \draw[->] (I)  to node[below] {$f$}      (J);
    \end{tikzpicture}

    By construction $P(ρ',f')$ is a $(J,κ)$-set for the extension $ρ' ∈ Γf'$ of $ρ$,
    and $af ∈ P(ι\,0)(f,ρ)$ is a $J$-element.
    We define $(P \op {ι} a)(f,ρ) ≔ \{ v \mid (af ,_κ v) ∈ P(ρ,f')\}$.

  \item[\sc In-Pred.]
    Let $f : I,ι → J$, $ρ : Γf$.  We need to define the $J$-set $(\ip {ι} A P)(f,ρ)$.

    \begin{itemize}
      \item If $f$ is not defined on $ι$, we have $f : I,ι \stackrel{(ι\,0)}{→} I \stackrel{g}{→} J$
        and we define $(\ip {ι} A P)(f,ρ) ≔ A(ρ',g)$
        where $ρ' ∈ Γg$ is the restriction of $ρ ∈ Γf$.

      \item Otherwise, there is a $κ ∈ J$ and $g : I → J_κ$ such that
        the following diagram commutes

        \begin{tikzpicture}[node distance=4\baselineskip]
          \node              (Ii) {$I,ι$};
          \node[below of=Ii] (I)  {$I$};
          \node[right of=Ii] (Jk) {$J$};
          \node[below of=Jk] (J)  {$J_κ$};

          \draw[->] (Ii) to node[left]  {$(ι\,0)$} (I);
          \draw[->] (Ii) to node[above] {$f$}      (Jk);
          \draw[->] (Jk) to node[right] {$(κ\,0)$} (J);
          \draw[->] (I)  to node[below] {$g$}      (J);
        \end{tikzpicture}

        We define $(\ip {ι} A P)(f,ρ) ≔ \{ (u ,_κ v) \mid u ∈ A(ρ',g), v ∈ \app{P(ρ',g)}u\}$
        where $ρ' ∈ Γg$ is the restriction of $ρ ∈ Γf$.
    \end{itemize}
\end{description}

\begin{theorem}
  If $Γ ⊢_I a : A$, $f : I → J$, $ρ ∈ Γf$, then $a(f,ρ) ∈ A(f,ρ)$.
  Furthermore $af(1,ρ) = a(f,ρ)$ (using the interpretation of
  $Γf ⊢_J af : Af$).
\end{theorem}

\begin{theorem}
  If $Γ ⊢_I a : A$ and $Γ ⊢_I b : A$ with $a = b$, then
  $a(f,ρ) = b(f,ρ)$ for any $f : I → J$ and $ρ : Γf$.
  In particular,
  \begin{itemize}
    \item $\app{λt}{u}(f,ρ) = t[u](f,ρ)$,
      where $[u]$ is the map $[u]ρ ↦ ⟨ρ,uρ⟩$, $Γ → Γ.A$
    \item $({(\ip {ι} A P)} \op {ι} a)(f,ρ) = (P\,a)(f,ρ)$
  \end{itemize}
\end{theorem}
\begin{proof}
  Let $f : I → J$, $ι ∉ I$, $κ = \fresh J$, $g : I,ι → J,κ$ such that
  $(ι\,0)f = g(κ\,0)$.
  We have
  \begin{align*}
    ({(\ip {ι} A P)} \op {ι} a)f
    &= \left\{ v \mid (af ,_κ v) ∈ (\ip {ι} A P)g \right\}
    \\
    &= \left\{ v \mid (af ,_κ v) ∈ \left\{ (x ,_κ y) \mid x ∈ Af, y ∈ \app{Pf} x \right\} \right\}
    \\
    &= \left\{ v \mid v ∈ \app{Pf}{af} \right\}
    \\
    &= \app{Pf}{af}
    \\
    &= \app{P}{a}f
  \qedhere
  \end{align*}
\end{proof}


\section{Related Work}

\paragraph{Our line of work}
This work continues a line of work aiming at a smooth integration of
parametricity with dependent types
\citep{bernardy_proofs_2012,bernardy_computational_2012,bernardy_type-theory_2013}. Compared to our 2013 publication, the present work offers two improvements:
1. a denotational semantics, and
2. a much simplified syntax, suitable as the basis of a proof assistant.

\paragraph{Parametric Models of Type-Theory vs Parametric Type-Theories}

Two pieces of work propose alternative parametric models of
type-theory
\citep{atkey_relationally_2014,krishnaswami_internalizing_2013}, but
do not integrate parametricity in the syntax of the calculus. This
means that, while certain consequences of parametricity can be made
available in the logic, via constants validated by the model,
parametricity itself is not available.


\paragraph{Various kinds of models}
Another characterising feature of proposals for parametricity is the
kind of model underlying the
semantics. \Citet{krishnaswami_internalizing_2013} propose a model
based on Q-PER. \Citet{atkey_relationally_2014} propose a model based
on reflexive graphs. The model that we use is based on cubes
(functions from subsets of colors). In
our 2012 work the cubes were reified as syntax in
an underlying calculus, while in the present work they refine a presheaf structure.

\paragraph{Cuboidal models}

We find remarkable that cuboidal models have been also successfully
used to interpret the univalence axiom.

Bezem, Coquand, Huber?
Pitts?

\bibliographystyle{abbrvnat}
\bibliography{PaperTools/bibtex/jp}

\end{document}
