\documentclass[10pt,a4paper]{article}
\usepackage[utf8]{inputenc}
\usepackage{amssymb,amstext,amsmath}
\usepackage{mathpartir}
\usepackage{mytheorems}
\usepackage[a4paper,margin=1.8cm]{geometry}

\newcommand\CC[4]{(#2,_{#1} #3)_{#4}}
\newcommand\CSig[1]{\Sigma^{#1}}
\newcommand\CTimes[3]{#2 ×_{#1} #3}
\newcommand\sw[2]{\mathsf{sw}^{#1}_{#2}}
\newcommand\dom{\mathsf{dom}}
\newcommand\param[1]{\!\cdot\!#1~}


\DeclareUnicodeCharacter{00A0}{~} %   NO-BREAK SPACE
\DeclareUnicodeCharacter{00A7}{\S} % §
\DeclareUnicodeCharacter{00AC}{\ensuremath{\neg}} % ¬
\DeclareUnicodeCharacter{00B0}{^{\circ}} % °
\DeclareUnicodeCharacter{00B1}{^1} 
\DeclareUnicodeCharacter{00B2}{^2} % ²
\DeclareUnicodeCharacter{00B7}{\ensuremath{\cdot}} % ·
\DeclareUnicodeCharacter{00B9}{\textsuperscript{l}} % ¹
\DeclareUnicodeCharacter{00D7}{\ensuremath{\times}} % × 
\DeclareUnicodeCharacter{00F7}{\ensuremath{\div}} % ÷
\DeclareUnicodeCharacter{02E1}{\ensuremath{{^l}}} % ˡ
\DeclareUnicodeCharacter{02B3}{\ensuremath{{^r}}} % ʳ
\DeclareUnicodeCharacter{0393}{\ensuremath{\Gamma}} % Γ
\DeclareUnicodeCharacter{0394}{\ensuremath{\Delta}} % Δ
\DeclareUnicodeCharacter{0397}{\ensuremath{\textrm{H}}} % Η
\DeclareUnicodeCharacter{0398}{\ensuremath{\Theta}} % Θ
\DeclareUnicodeCharacter{039B}{\ensuremath{\Lambda}} % Λ
\DeclareUnicodeCharacter{039E}{\ensuremath{\Xi}} % Ξ
\DeclareUnicodeCharacter{03A3}{\ensuremath{\Sigma}} % Σ
\DeclareUnicodeCharacter{03A6}{\ensuremath{\Phi}} % Φ
\DeclareUnicodeCharacter{03A8}{\ensuremath{\Psi}} % Ψ
\DeclareUnicodeCharacter{03A9}{\ensuremath{\Omega}} % Ω
\DeclareUnicodeCharacter{03B1}{\ensuremath{\mathnormal{\alpha}}} % α
\DeclareUnicodeCharacter{03B2}{\ensuremath{\beta}} % β
\DeclareUnicodeCharacter{03B3}{\ensuremath{\mathnormal{\gamma}}} % γ
\DeclareUnicodeCharacter{03B4}{\ensuremath{\mathnormal{\delta}}} % δ
\DeclareUnicodeCharacter{03B5}{\ensuremath{\mathnormal{\varepsilon}}} % ε
\DeclareUnicodeCharacter{03B6}{\ensuremath{\mathnormal{\zeta}}} % ζ
\DeclareUnicodeCharacter{03B7}{\ensuremath{\mathnormal{\eta}}} % η
\DeclareUnicodeCharacter{03B8}{\ensuremath{\mathnormal{\theta}}} % θ
\DeclareUnicodeCharacter{03B9}{\ensuremath{\mathnormal{\iota}}} % ι
\DeclareUnicodeCharacter{03BA}{\ensuremath{\mathnormal{\kappa}}} % κ
\DeclareUnicodeCharacter{03BB}{\ensuremath{\mathnormal{\lambda}}} % λ
\DeclareUnicodeCharacter{03BC}{\ensuremath{\mathnormal{\mu}}} % μ
\DeclareUnicodeCharacter{03BD}{\ensuremath{\mathnormal{\mu}}} % ν
\DeclareUnicodeCharacter{03BE}{\ensuremath{\mathnormal{\xi}}} % ξ
\DeclareUnicodeCharacter{03C0}{\ensuremath{\mathnormal{\pi}}} % π
\DeclareUnicodeCharacter{03C1}{\ensuremath{\mathnormal{\rho}}} % ρ
\DeclareUnicodeCharacter{03C3}{\ensuremath{\mathnormal{\sigma}}} % σ
\DeclareUnicodeCharacter{03C4}{\ensuremath{\mathnormal{\tau}}} % τ
\DeclareUnicodeCharacter{03C6}{\ensuremath{\mathnormal{\varphi}}} % φ
\DeclareUnicodeCharacter{03D5}{\ensuremath{\mathnormal{\phi}}} % ϕ
\DeclareUnicodeCharacter{03C7}{\ensuremath{\mathnormal{\chi}}} % χ
\DeclareUnicodeCharacter{03C8}{\ensuremath{\mathnormal{\psi}}} % ψ
\DeclareUnicodeCharacter{03C9}{\ensuremath{\mathnormal{\omega}}} % ω 
\DeclareUnicodeCharacter{03F5}{\ensuremath{\mathnormal{\epsilon}}} % ϵ
\DeclareUnicodeCharacter{1D62}{_i} % ᵢ
\DeclareUnicodeCharacter{10627}{\ensuremath{\lbana}} 
\DeclareUnicodeCharacter{10628}{\ensuremath{\rbana}} 
\DeclareUnicodeCharacter{2026}{\ensuremath{\ldots}}
\DeclareUnicodeCharacter{202F}{{\,}}
\DeclareUnicodeCharacter{2080}{\ensuremath{_0}} % ₀
\DeclareUnicodeCharacter{2081}{\ensuremath{_1}}
\DeclareUnicodeCharacter{2082}{\ensuremath{_2}}
\DeclareUnicodeCharacter{2083}{\ensuremath{_3}}
\DeclareUnicodeCharacter{2084}{\ensuremath{_4}}
\DeclareUnicodeCharacter{2085}{\ensuremath{_5}}
\DeclareUnicodeCharacter{2086}{\ensuremath{_6}}
\DeclareUnicodeCharacter{2087}{\ensuremath{_7}}
\DeclareUnicodeCharacter{2088}{\ensuremath{_8}}
\DeclareUnicodeCharacter{2089}{\ensuremath{_9}}
\DeclareUnicodeCharacter{2115}{\mathbb{N}}
\DeclareUnicodeCharacter{214B}{\ensuremath{\parr}}
\DeclareUnicodeCharacter{2190}{\ensuremath{\leftarrow}} % ← 
\DeclareUnicodeCharacter{2191}{\ensuremath{\uparrow}} % ↑
\DeclareUnicodeCharacter{2192}{\ensuremath{\rightarrow}} % →
\DeclareUnicodeCharacter{2194}{\ensuremath{\leftrightarrow}} % ↔
\DeclareUnicodeCharacter{2196}{\nwarrow} % ↖
\DeclareUnicodeCharacter{2197}{\nearrow} % ↗
\DeclareUnicodeCharacter{219D}{\ensuremath{\leadsto}} % ↝
\DeclareUnicodeCharacter{21A6}{\ensuremath{\mapsto}} % ↦ 
\DeclareUnicodeCharacter{21C6}{\ensuremath{\leftrightarrows}} % ⇆
\DeclareUnicodeCharacter{21D0}{\ensuremath{\Leftarrow}} % ⇐
\DeclareUnicodeCharacter{21D2}{\ensuremath{\Rightarrow}} % ⇒ 
\DeclareUnicodeCharacter{21D4}{\ensuremath{\Leftrightarrow}} % ⇔
\DeclareUnicodeCharacter{2200}{\ensuremath{\forall}} % ∀
\DeclareUnicodeCharacter{2203}{\ensuremath{\exists}} % ∃
\DeclareUnicodeCharacter{2205}{\ensuremath{\varnothing}} % ∅
\DeclareUnicodeCharacter{2208}{\ensuremath{\in}} % ∈
\DeclareUnicodeCharacter{2209}{\ensuremath{\not\in}} % ∉
\DeclareUnicodeCharacter{220B}{\ensuremath{\ni}}
\DeclareUnicodeCharacter{220E}{\ensuremath{\qed}} % ∎ % Alternatively use \blacksquare
\DeclareUnicodeCharacter{2211}{\sum}% ∑
\DeclareUnicodeCharacter{2215}{\mathbb{N}} % ℕ
\DeclareUnicodeCharacter{2217}{\ensuremath{\ast}} % ∗
\DeclareUnicodeCharacter{2218}{\ensuremath{\circ}} % ∘
\DeclareUnicodeCharacter{2219}{\ensuremath{\bullet}} % ∙ 
\DeclareUnicodeCharacter{221E}{\ensuremath{\infty}} % ∞
\DeclareUnicodeCharacter{2223}{\ensuremath{\mid}} % ∣
\DeclareUnicodeCharacter{2227}{\wedge}% ∧
\DeclareUnicodeCharacter{2228}{\vee}% ∨
\DeclareUnicodeCharacter{2229}{\ensuremath{\cap}} % ∩
\DeclareUnicodeCharacter{222A}{\ensuremath{\cup}} % ∪
\DeclareUnicodeCharacter{2237}{::} % ∷
\DeclareUnicodeCharacter{223C}{\ensuremath{\sim}} % ∼
\DeclareUnicodeCharacter{2243}{\ensuremath{\simeq}} % ≃
\DeclareUnicodeCharacter{2245}{\ensuremath{\cong}} % ≅ 
\DeclareUnicodeCharacter{2248}{\ensuremath{\approx}} % ≈
\DeclareUnicodeCharacter{225C}{\ensuremath{\stackrel{\scriptscriptstyle {\triangle}}{=}}} % ≜
\DeclareUnicodeCharacter{225F}{\ensuremath{\stackrel{\scriptscriptstyle ?}{=}}} % ≟
\DeclareUnicodeCharacter{2260}{\neq}% ≠
\DeclareUnicodeCharacter{2261}{\equiv}% ≡
\DeclareUnicodeCharacter{2264}{\ensuremath{\le}} % ≤
\DeclareUnicodeCharacter{2265}{\ensuremath{\ge}} % ≥
\DeclareUnicodeCharacter{2282}{\ensuremath{\subset}} % ⊂
\DeclareUnicodeCharacter{2283}{\ensuremath{\supset}} % ⊃ 
\DeclareUnicodeCharacter{2286}{\ensuremath{\subseteq}} % ⊆ 
\DeclareUnicodeCharacter{2287}{\ensuremath{\supseteq}} % ⊇
\DeclareUnicodeCharacter{2293}{\ensuremath{\sqcup}} % ⊓
\DeclareUnicodeCharacter{2293}{\sqcap} % ⊓
\DeclareUnicodeCharacter{2294}{\sqcup} % ⊔
\DeclareUnicodeCharacter{2295}{\ensuremath{\oplus}} % ⊕
\DeclareUnicodeCharacter{2297}{\ensuremath{\otimes}} % ⊗
\DeclareUnicodeCharacter{22A2}{\ensuremath{\vdash}}
\DeclareUnicodeCharacter{22A4}{\ensuremath{\top}} % ⊤
\DeclareUnicodeCharacter{22A5}{\ensuremath{\bot}} % ⊥
\DeclareUnicodeCharacter{22A7}{\models} % ⊧ 
\DeclareUnicodeCharacter{22A8}{\models} % ⊨
\DeclareUnicodeCharacter{22A9}{\Vdash} % ⊩
\DeclareUnicodeCharacter{22B8}{\ensuremath{\multimap}} % ⊸
\DeclareUnicodeCharacter{22C4}{\diamond} % ⋄
\DeclareUnicodeCharacter{22C6}{\ensuremath{\star}}
\DeclareUnicodeCharacter{22EE}{\ensuremath{\vdots}} % ⋮
\DeclareUnicodeCharacter{22EF}{\ensuremath{\cdots}} % ⋯
\DeclareUnicodeCharacter{2308}{\ensuremath{\lceil}}
\DeclareUnicodeCharacter{2309}{\ensuremath{\rceil}}
\DeclareUnicodeCharacter{230A}{\ensuremath{\lfloor}}
\DeclareUnicodeCharacter{230B}{\ensuremath{\rfloor}}
\DeclareUnicodeCharacter{25A1}{\ensuremath{\square}} % □
\DeclareUnicodeCharacter{25AF}{\mathop{\talloblong}} % ▯
\DeclareUnicodeCharacter{25C7}{\diamond} % ◇
\DeclareUnicodeCharacter{2605}{\ensuremath{\star}}   % ★
\DeclareUnicodeCharacter{2713}{\ensuremath{\checkmark}} % ✓
\DeclareUnicodeCharacter{27C2}{\ensuremath{^\bot}} % PERPENDICULAR ⟂
\DeclareUnicodeCharacter{27E6}{\ensuremath{\llbracket}} % ⟦
\DeclareUnicodeCharacter{27E7}{\ensuremath{\rrbracket}} % ⟧
\DeclareUnicodeCharacter{27E8}{\ensuremath{\langle}} % ⟨
\DeclareUnicodeCharacter{27E9}{\ensuremath{\rangle}} % ⟩
\DeclareUnicodeCharacter{27F6}{{\longrightarrow}} % ⟶
\DeclareUnicodeCharacter{27F7}{{\longleftrightarrow}} % ⟷
\DeclareUnicodeCharacter{2A04}{\mathop{\dot{\cup}}} % ⨄
\DeclareUnicodeCharacter{2AFE}{\mathop{\talloblong}} % ⫾

% \DeclareUnicodeCharacter{8499}{\mathcal{M}} 
% \DeclareUnicodeCharacter{8718}{\ensuremath{\blacksquare}}
% \DeclareUnicodeCharacter{8797}{\mathrel{\mathop:}=}
% \DeclareUnicodeCharacter{9657}{\ensuremath{\triangleright}}
% \DeclareUnicodeCharacter{9667}{\triangleright{}}
% \DeclareUnicodeCharacter{9669}{\ensuremath{\triangleleft}}


\title{A parametric Type theory}

\author{}
\date{\today}

\begin{document}
\maketitle


\begin{definition}[Syntax of terms]
  
  \begin{align*}
    ρ,σ & ::= permutations \\
    t,u,A,B & ::= x ~|~ U \\
            & ~|~ λx:A. t      ~|~ t u ~|~ (x:A) → B \\
            & ~|~ \CC i t u A  ~|~ t \param i ~|~ \sw {σ} T
  \end{align*}
\end{definition}
$\CSig i A P$ and $(x:A) ×_i P x$ are other notations for $(A ,_i P)_U$.

\begin{definition}[Typing relation]
Contexts:
  \begin{mathpar}
    \inferrule[Empty]{~}{ ⊢_I }

    \inferrule[Cons]{Γ⊢_I \\ Γ ⊢_I A }{ Γ,A ⊢_I }
  \end{mathpar}
(Could be removed in presence of weakening; except Param rules uses it.)
 
Types:
 \begin{mathpar}
    \inferrule[Universe]{Γ ⊢_I}{Γ ⊢_I U : U}

    \inferrule[Pi]{Γ ⊢_I A : U \\ Γ,x:A ⊢_I B : U}{Γ ⊢_I (x:A) → B : U}

  \end{mathpar}

Terms:
  \begin{mathpar}
    \inferrule[Var]{x:A ∈ Γ \\Γ ⊢_I A}{Γ ⊢_I x : A}

    \inferrule[Wk]{Γ ⊢_I A \\Γ ⊢_I t : T}{Γ,x:A ⊢_I t : T}

    \inferrule[Lam]{Γ,x:A ⊢_I B}{Γ ⊢_I λx:A.b : (x:A) → B}
 
    \inferrule[App]{Γ ⊢_I t : (x:A) → B \\ Γ ⊢_I u : A}{Γ ⊢_I t u: B[u/x] }

    \\\inferrule[Coloring]{Γ ⊢_I \\Γ ⊢_I a : T i0 \\Γ ⊢_I p : T.i a}{Γ ⊢_{I,i} (a,_i p)_T : T}

    \inferrule[Param]{Γ ⊢_I \\Γ ⊢_{I,i} a : T}{Γ ⊢_I a \param i : T \param i a (i0)}
 
    \inferrule[Swap]{\dom (σ) = J \\ I ∩ J = ∅ \\ Γ ⊢_{I} T }{Γ ⊢_{I} \sw {σ} T : (x : T)_J → T ∋_{(σ J)} x  }

  \end{mathpar}
\end{definition}

where the hypercubic expansions used in the swap rule are defined as follows:

\begin{align*}
(x:T)_{∅} & = x:T \\
(x:T)_{i,J} & = (x:T(i0))_J, (x_i:T.i x)_J
\end{align*}
\begin{align*}
T ∋_{∅} x &= T \\
T ∋_{J,k} x & = ((T ∋_J x) [({x_I},_k {x_{I,k}}) /x \mid I ⊂ J]).k   x_J
\end{align*}


\begin{definition}[Reduction]~

Functions:
\begin{align*}
  (λx:A. u[x]) t &= u[t]  \\
  (f ,_i g) a & = (f a(i0) ,_i g a(i0) a.i)
\end{align*}

Param:
\begin{align*}
  U.i T &= T → U \\
  ((x:A) → B x).i f &= (x:A) → (x' : A.i x) → (B (x,_i x')).i (f x) \\
  (λx. b[x]).i &= (λx:A. λx':A.i x. b[\CC i x {x'} A].i) \\
  (f a).i &= f.i (a (i0)) (a.i) \\
  (a,_i p)_T.i  &= p \\
  (a,_i p)_T.j  &= (a.j ,_i \sw {i j} T   a(j0)   a.j   p(j0)   p.j)_{T.j ((a ,_i p) j0)}  & (*) \\
    (\sw {σ} T).i  &= \sw {σ,i↦i} T \\ 
\end{align*}

Swap: We have two sets of reduction rules for swap. If the type is canonical, then the substitution is expanded into a series of swaps. Each swap can then be expanded until one hits a neutral term.
Swaps where T is neutral are on the other hand composed.
\begin{align*}
  \sw {σ} n ⋯ (\sw {ρ} n ⋯ t)  &= \sw {σ ∘ ρ} n ⋯ t & (**)\\
  \sw {(ij)∘σ} T & = \text{expand if $T$ is canonical.}\\
  \sw {i j} {U} A A_j A_i A_{ij} a a_i a_j & = A_{ij} a a_j a_i  \\
  \sw {i j} {(x:A) → B[x]} f f_j f_i f_{ij} a a_i a_j a_{ji} & = \sw {i j} {B[((a,_ia_i),_j(a_j,_ia_{ji}))]} (f a)
(f_j a a_j) (f_i a a_i) (f_{ij} a a_j a_i (\sw {j i} A a a_i a_j a_{ji})) \\
  \sw {i j} {(A ,_i P)_U} x x_j x_i x_{ij} & = x_{ij} \\
  \sw {i j} {(A ,_j P)_U} x x_j x_i x_{ij} & = x_{ij} \\
  \sw {i j} {(A ,_k P)_U} (x ,_k x_k) (x_j ,_k x_{jk}) (x_i ,_k x_{ik}) (x_{ij},_k x_{ijk}) & = 
    (\sw {i j} A x x_j x_i x_{ij},_k \sw {i j} {P ((x,_jx_j),_i(x_i,_j x_{ij}))} x_k x_{jk} x_{jk} x_{ijk})_X \\
\end{align*}


\begin{itemize}
\item 
(*): $p : T.i a$ so $p.j : (T.i a).j p(j0)$. But,
the second component of the pair should be of type $(T.j ((a ,_i p)
j0)).i a.j$.  That is, we need $\sw{ i j} T$ to be a conversion from
$T.i.j a(j0) a.j p(j0)$ to $T.j.i a(j0) p(j0) a.j$.

\item (**): $⋯$ stands for $2^{|σ|}-1$ arguments. $n$ is neutral. Both
  sequences are guaranteed to be consistent (one is a swap of the
  other). In fact, they are inferrable from the type of the last
  argument, because $n$ is a neutral term.
\item $ij$ stands for the substitution $ij ↦ ji$.
\item (***): The source type of the swap is
$$W = (x_{ij}:A.i.j x x_j x_i) \times_k P.i.j x x_j x_i x_{ij} x_k x_{jk}
x_{ik}$$ By computation (using other cases of the definition), all the
occurences of swap disappear in $W$.

The target type is $X$ and $X = (x_{ji}:A.j.i x x_i x_j) \times_k
P.j.i x x_i x_j x_{ji} x_k x_{ik} x_{jk}$.  One might think that swaps
should be applied to certain variables in $X$, but it is not
the case.  This can be checked by verifying that the following types
for variables match both the source and the target.

\end{itemize}
\end{definition}

\begin{definition}[Conversion]
  \begin{mathpar}
    \inferrule{t (i0) = a \\ t.i = p} {t = \CC i a p T}
  \end{mathpar}
+ congruences.
\end{definition}




\end{document}
