\documentclass[10pt,a4paper]{article}
\usepackage[utf8]{inputenc}
\usepackage{amssymb,amstext,amsmath}
\usepackage{mathpartir}
\usepackage{mytheorems}
\usepackage[a4paper,margin=1.8cm]{geometry}
\usepackage{fancyref}

\newcommand\CC[4]{(#2,_{#1} #3)_{#4}}
\newcommand\CSig[1]{\Sigma^{#1}}
\newcommand\CTimes[3]{#2 ×_{#1} #3}
\newcommand\SW[2]{\mathsf{SW}^{#1}_{#2}}
\newcommand\sw[2]{\mathsf{sw}^{#1}_{#2}}
\newcommand\dom{\mathsf{dom}}
\newcommand\param[1]{\!\cdot\!#1~}


\DeclareUnicodeCharacter{00A0}{~} %   NO-BREAK SPACE
\DeclareUnicodeCharacter{00A7}{\S} % §
\DeclareUnicodeCharacter{00AC}{\ensuremath{\neg}} % ¬
\DeclareUnicodeCharacter{00B0}{^{\circ}} % °
\DeclareUnicodeCharacter{00B1}{^1} 
\DeclareUnicodeCharacter{00B2}{^2} % ²
\DeclareUnicodeCharacter{00B7}{\ensuremath{\cdot}} % ·
\DeclareUnicodeCharacter{00B9}{\textsuperscript{l}} % ¹
\DeclareUnicodeCharacter{00D7}{\ensuremath{\times}} % × 
\DeclareUnicodeCharacter{00F7}{\ensuremath{\div}} % ÷
\DeclareUnicodeCharacter{02E1}{\ensuremath{{^l}}} % ˡ
\DeclareUnicodeCharacter{02B3}{\ensuremath{{^r}}} % ʳ
\DeclareUnicodeCharacter{0393}{\ensuremath{\Gamma}} % Γ
\DeclareUnicodeCharacter{0394}{\ensuremath{\Delta}} % Δ
\DeclareUnicodeCharacter{0397}{\ensuremath{\textrm{H}}} % Η
\DeclareUnicodeCharacter{0398}{\ensuremath{\Theta}} % Θ
\DeclareUnicodeCharacter{039B}{\ensuremath{\Lambda}} % Λ
\DeclareUnicodeCharacter{039E}{\ensuremath{\Xi}} % Ξ
\DeclareUnicodeCharacter{03A3}{\ensuremath{\Sigma}} % Σ
\DeclareUnicodeCharacter{03A6}{\ensuremath{\Phi}} % Φ
\DeclareUnicodeCharacter{03A8}{\ensuremath{\Psi}} % Ψ
\DeclareUnicodeCharacter{03A9}{\ensuremath{\Omega}} % Ω
\DeclareUnicodeCharacter{03B1}{\ensuremath{\mathnormal{\alpha}}} % α
\DeclareUnicodeCharacter{03B2}{\ensuremath{\beta}} % β
\DeclareUnicodeCharacter{03B3}{\ensuremath{\mathnormal{\gamma}}} % γ
\DeclareUnicodeCharacter{03B4}{\ensuremath{\mathnormal{\delta}}} % δ
\DeclareUnicodeCharacter{03B5}{\ensuremath{\mathnormal{\varepsilon}}} % ε
\DeclareUnicodeCharacter{03B6}{\ensuremath{\mathnormal{\zeta}}} % ζ
\DeclareUnicodeCharacter{03B7}{\ensuremath{\mathnormal{\eta}}} % η
\DeclareUnicodeCharacter{03B8}{\ensuremath{\mathnormal{\theta}}} % θ
\DeclareUnicodeCharacter{03B9}{\ensuremath{\mathnormal{\iota}}} % ι
\DeclareUnicodeCharacter{03BA}{\ensuremath{\mathnormal{\kappa}}} % κ
\DeclareUnicodeCharacter{03BB}{\ensuremath{\mathnormal{\lambda}}} % λ
\DeclareUnicodeCharacter{03BC}{\ensuremath{\mathnormal{\mu}}} % μ
\DeclareUnicodeCharacter{03BD}{\ensuremath{\mathnormal{\mu}}} % ν
\DeclareUnicodeCharacter{03BE}{\ensuremath{\mathnormal{\xi}}} % ξ
\DeclareUnicodeCharacter{03C0}{\ensuremath{\mathnormal{\pi}}} % π
\DeclareUnicodeCharacter{03C1}{\ensuremath{\mathnormal{\rho}}} % ρ
\DeclareUnicodeCharacter{03C3}{\ensuremath{\mathnormal{\sigma}}} % σ
\DeclareUnicodeCharacter{03C4}{\ensuremath{\mathnormal{\tau}}} % τ
\DeclareUnicodeCharacter{03C6}{\ensuremath{\mathnormal{\varphi}}} % φ
\DeclareUnicodeCharacter{03D5}{\ensuremath{\mathnormal{\phi}}} % ϕ
\DeclareUnicodeCharacter{03C7}{\ensuremath{\mathnormal{\chi}}} % χ
\DeclareUnicodeCharacter{03C8}{\ensuremath{\mathnormal{\psi}}} % ψ
\DeclareUnicodeCharacter{03C9}{\ensuremath{\mathnormal{\omega}}} % ω 
\DeclareUnicodeCharacter{03F5}{\ensuremath{\mathnormal{\epsilon}}} % ϵ
\DeclareUnicodeCharacter{1D62}{_i} % ᵢ
\DeclareUnicodeCharacter{10627}{\ensuremath{\lbana}} 
\DeclareUnicodeCharacter{10628}{\ensuremath{\rbana}} 
\DeclareUnicodeCharacter{2026}{\ensuremath{\ldots}}
\DeclareUnicodeCharacter{202F}{{\,}}
\DeclareUnicodeCharacter{2080}{\ensuremath{_0}} % ₀
\DeclareUnicodeCharacter{2081}{\ensuremath{_1}}
\DeclareUnicodeCharacter{2082}{\ensuremath{_2}}
\DeclareUnicodeCharacter{2083}{\ensuremath{_3}}
\DeclareUnicodeCharacter{2084}{\ensuremath{_4}}
\DeclareUnicodeCharacter{2085}{\ensuremath{_5}}
\DeclareUnicodeCharacter{2086}{\ensuremath{_6}}
\DeclareUnicodeCharacter{2087}{\ensuremath{_7}}
\DeclareUnicodeCharacter{2088}{\ensuremath{_8}}
\DeclareUnicodeCharacter{2089}{\ensuremath{_9}}
\DeclareUnicodeCharacter{2115}{\mathbb{N}}
\DeclareUnicodeCharacter{214B}{\ensuremath{\parr}}
\DeclareUnicodeCharacter{2190}{\ensuremath{\leftarrow}} % ← 
\DeclareUnicodeCharacter{2191}{\ensuremath{\uparrow}} % ↑
\DeclareUnicodeCharacter{2192}{\ensuremath{\rightarrow}} % →
\DeclareUnicodeCharacter{2194}{\ensuremath{\leftrightarrow}} % ↔
\DeclareUnicodeCharacter{2196}{\nwarrow} % ↖
\DeclareUnicodeCharacter{2197}{\nearrow} % ↗
\DeclareUnicodeCharacter{219D}{\ensuremath{\leadsto}} % ↝
\DeclareUnicodeCharacter{21A6}{\ensuremath{\mapsto}} % ↦ 
\DeclareUnicodeCharacter{21C6}{\ensuremath{\leftrightarrows}} % ⇆
\DeclareUnicodeCharacter{21D0}{\ensuremath{\Leftarrow}} % ⇐
\DeclareUnicodeCharacter{21D2}{\ensuremath{\Rightarrow}} % ⇒ 
\DeclareUnicodeCharacter{21D4}{\ensuremath{\Leftrightarrow}} % ⇔
\DeclareUnicodeCharacter{2200}{\ensuremath{\forall}} % ∀
\DeclareUnicodeCharacter{2203}{\ensuremath{\exists}} % ∃
\DeclareUnicodeCharacter{2205}{\ensuremath{\varnothing}} % ∅
\DeclareUnicodeCharacter{2208}{\ensuremath{\in}} % ∈
\DeclareUnicodeCharacter{2209}{\ensuremath{\not\in}} % ∉
\DeclareUnicodeCharacter{220B}{\ensuremath{\ni}}
\DeclareUnicodeCharacter{220E}{\ensuremath{\qed}} % ∎ % Alternatively use \blacksquare
\DeclareUnicodeCharacter{2211}{\sum}% ∑
\DeclareUnicodeCharacter{2215}{\mathbb{N}} % ℕ
\DeclareUnicodeCharacter{2217}{\ensuremath{\ast}} % ∗
\DeclareUnicodeCharacter{2218}{\ensuremath{\circ}} % ∘
\DeclareUnicodeCharacter{2219}{\ensuremath{\bullet}} % ∙ 
\DeclareUnicodeCharacter{221E}{\ensuremath{\infty}} % ∞
\DeclareUnicodeCharacter{2223}{\ensuremath{\mid}} % ∣
\DeclareUnicodeCharacter{2227}{\wedge}% ∧
\DeclareUnicodeCharacter{2228}{\vee}% ∨
\DeclareUnicodeCharacter{2229}{\ensuremath{\cap}} % ∩
\DeclareUnicodeCharacter{222A}{\ensuremath{\cup}} % ∪
\DeclareUnicodeCharacter{2237}{::} % ∷
\DeclareUnicodeCharacter{223C}{\ensuremath{\sim}} % ∼
\DeclareUnicodeCharacter{2243}{\ensuremath{\simeq}} % ≃
\DeclareUnicodeCharacter{2245}{\ensuremath{\cong}} % ≅ 
\DeclareUnicodeCharacter{2248}{\ensuremath{\approx}} % ≈
\DeclareUnicodeCharacter{225C}{\ensuremath{\stackrel{\scriptscriptstyle {\triangle}}{=}}} % ≜
\DeclareUnicodeCharacter{225F}{\ensuremath{\stackrel{\scriptscriptstyle ?}{=}}} % ≟
\DeclareUnicodeCharacter{2260}{\neq}% ≠
\DeclareUnicodeCharacter{2261}{\equiv}% ≡
\DeclareUnicodeCharacter{2264}{\ensuremath{\le}} % ≤
\DeclareUnicodeCharacter{2265}{\ensuremath{\ge}} % ≥
\DeclareUnicodeCharacter{2282}{\ensuremath{\subset}} % ⊂
\DeclareUnicodeCharacter{2283}{\ensuremath{\supset}} % ⊃ 
\DeclareUnicodeCharacter{2286}{\ensuremath{\subseteq}} % ⊆ 
\DeclareUnicodeCharacter{2287}{\ensuremath{\supseteq}} % ⊇
\DeclareUnicodeCharacter{2293}{\ensuremath{\sqcup}} % ⊓
\DeclareUnicodeCharacter{2293}{\sqcap} % ⊓
\DeclareUnicodeCharacter{2294}{\sqcup} % ⊔
\DeclareUnicodeCharacter{2295}{\ensuremath{\oplus}} % ⊕
\DeclareUnicodeCharacter{2297}{\ensuremath{\otimes}} % ⊗
\DeclareUnicodeCharacter{22A2}{\ensuremath{\vdash}}
\DeclareUnicodeCharacter{22A4}{\ensuremath{\top}} % ⊤
\DeclareUnicodeCharacter{22A5}{\ensuremath{\bot}} % ⊥
\DeclareUnicodeCharacter{22A7}{\models} % ⊧ 
\DeclareUnicodeCharacter{22A8}{\models} % ⊨
\DeclareUnicodeCharacter{22A9}{\Vdash} % ⊩
\DeclareUnicodeCharacter{22B8}{\ensuremath{\multimap}} % ⊸
\DeclareUnicodeCharacter{22C4}{\diamond} % ⋄
\DeclareUnicodeCharacter{22C6}{\ensuremath{\star}}
\DeclareUnicodeCharacter{22EE}{\ensuremath{\vdots}} % ⋮
\DeclareUnicodeCharacter{22EF}{\ensuremath{\cdots}} % ⋯
\DeclareUnicodeCharacter{2308}{\ensuremath{\lceil}}
\DeclareUnicodeCharacter{2309}{\ensuremath{\rceil}}
\DeclareUnicodeCharacter{230A}{\ensuremath{\lfloor}}
\DeclareUnicodeCharacter{230B}{\ensuremath{\rfloor}}
\DeclareUnicodeCharacter{25A1}{\ensuremath{\square}} % □
\DeclareUnicodeCharacter{25AF}{\mathop{\talloblong}} % ▯
\DeclareUnicodeCharacter{25C7}{\diamond} % ◇
\DeclareUnicodeCharacter{2605}{\ensuremath{\star}}   % ★
\DeclareUnicodeCharacter{2713}{\ensuremath{\checkmark}} % ✓
\DeclareUnicodeCharacter{27C2}{\ensuremath{^\bot}} % PERPENDICULAR ⟂
\DeclareUnicodeCharacter{27E6}{\ensuremath{\llbracket}} % ⟦
\DeclareUnicodeCharacter{27E7}{\ensuremath{\rrbracket}} % ⟧
\DeclareUnicodeCharacter{27E8}{\ensuremath{\langle}} % ⟨
\DeclareUnicodeCharacter{27E9}{\ensuremath{\rangle}} % ⟩
\DeclareUnicodeCharacter{27F6}{{\longrightarrow}} % ⟶
\DeclareUnicodeCharacter{27F7}{{\longleftrightarrow}} % ⟷
\DeclareUnicodeCharacter{2A04}{\mathop{\dot{\cup}}} % ⨄
\DeclareUnicodeCharacter{2AFE}{\mathop{\talloblong}} % ⫾

% \DeclareUnicodeCharacter{8499}{\mathcal{M}} 
% \DeclareUnicodeCharacter{8718}{\ensuremath{\blacksquare}}
% \DeclareUnicodeCharacter{8797}{\mathrel{\mathop:}=}
% \DeclareUnicodeCharacter{9657}{\ensuremath{\triangleright}}
% \DeclareUnicodeCharacter{9667}{\triangleright{}}
% \DeclareUnicodeCharacter{9669}{\ensuremath{\triangleleft}}


\title{A parametric Type theory}

\author{}
\date{\today}

\begin{document}
\maketitle


\begin{definition}[Syntax of terms and contexts]
  
  \begin{align*}
    t,u,A,B & ::= x ~|~ U ~|~ λx:A. t      ~|~ t u ~|~ (x:A) → B \\
            & ~|~ \CC i t u A  ~|~ t \param i ~|~ \sw {ij} T \\
    \Gamma,\Delta & ::= () ~|~ \Gamma,x:A ~|~ \Gamma,I 
  \end{align*}
\end{definition}
$\CSig i A P$ and $(x:A) ×_i P x$ are other notations for $(A ,_i P)_U$.

 We use $I,J,…$ for finite {\em sets} of colours (order does not matter).

\begin{definition}[Typing relation]
Contexts:
  \begin{mathpar}
    \inferrule[Empty]{~}{() ⊢ }

    \inferrule[Cons]{Γ⊢ \\ Γ ⊢ A }{ Γ,x:A ⊢ }

    \inferrule[CCons]{Γ⊢ \\ Γ ⊢ }{ Γ,I ⊢ }

    \inferrule[NewVar]{Γ⊢ \\ Γ ⊢ A }{ Γ,x:A ⊢ }

    \inferrule[NewColor]{Γ⊢}{ Γ,I ⊢ }

  \end{mathpar}
 
Types:
 \begin{mathpar}
    \inferrule[Universe]{Γ ⊢}{Γ ⊢ U : U}

    \inferrule[Pi]{Γ ⊢ A : U \\ Γ,x:A ⊢ B : U}{Γ ⊢ (x:A) → B : U}

  \end{mathpar}

Terms:
  \begin{mathpar}
    \inferrule[Conv]{Γ ⊢ t:A \\ A = B}{Γ ⊢ t : B}

    \inferrule[Var]{Γ ⊢ A}{Γ,x:A ⊢ x : A}

    \inferrule[Wk]{Γ ⊢ A \\Γ ⊢ t : T}{Γ,x:A ⊢ t : T}

    \inferrule[CWk]{Γ ⊢ t : T}{Γ,I ⊢ t : T}

    \\\inferrule[Lam]{Γ,x:A ⊢ B}{Γ ⊢ λx:A.b : (x:A) → B}
 
    \inferrule[App]{Γ ⊢ t : (x:A) → B[x] \\ Γ ⊢ u : A}{Γ ⊢ t u: B[u] }

    \\\inferrule[Coloring]{Γ,I ⊢ a : T i0 \\Γ,I ⊢ p : T \param i a}{Γ,I,i ⊢ (a,_i p)_T : T}

    \inferrule[Param]{Γ,I,i ⊢ a : T}{Γ,I ⊢ a \param i : T \param i a (i0)}
 
    \inferrule[Swap]{Γ,I,i,j ⊢ T }{Γ,I ⊢ \sw {ij} T : \SW {ij} T}

  \end{mathpar}
\end{definition}
%    \inferrule[Swap]{\dom (σ) = J \\ I ∩ J = ∅ \\ Γ ⊢_{I} T }{Γ ⊢_{I} \sw {σ} T : (x : T)_J → T ∋_{(σ J)} x  }


where $\SW {ij} T = (x:T(i0)(j0)) → (x_j:T(i0) \param j x) → (x_i : T(j0) \param i x) → T \param i \param j x x_j x_i → T \param j \param i x x_i x_j$.

\begin{definition}[Reduction]~

Functions:
\begin{align*}
  (λx:A. u[x]) t &= u[t]  \\
  (f ,_i g)_{(x:A)→ B[x]} a & = (f a(i0) ,_i g a(i0) a \param i)_{B[a]}
\end{align*}

Param:
\begin{align*}
  U\param i T &= T → U \\
  ((x:A) → B[x])\param i f &= (x:A) → (x' : A\param i x) → (B[(x,_i x')_A])\param i (f x) \\
  (λx:A. b[x])\param i &= λx:A(i0). λx':A\param i x. b[\CC i x {x'} A]\param i \\
  (f a)\param i &= f\param i (a (i0)) (a\param i) \\
  (a,_i p)_T\param i  &= p \\
  (a,_i p)_T \param j  &= (a \param j ,_i \sw {i j} T   a(j0)   a \param j   p(j0)   p \param j)_{T \param j ((a ,_i p) j0)}  & (*)
\end{align*}

Swap:
\begin{align*}
  \sw {i j} {U} A A_j A_i A_{ij} a a_i a_j & = A_{ij} a a_j a_i  \\
  \sw {i j} {(x:A) → B[x]} f f_j f_i f_{ij} a a_i a_j a_{ji} & = \sw {i j} {B[((a,_ia_i),_j(a_j,_ia_{ji}))]} (f a)
(f_j a a_j) (f_i a a_i) (f_{ij} a a_j a_i (\sw {j i} A a a_i a_j a_{ji})) \\
  \sw {i j} {(A ,_i P)_U} x x_j x_i x_{ij} & = x_{ij} \\
  \sw {i j} {(A ,_j P)_U} x x_j x_i x_{ij} & = x_{ij} \\
  \sw {i j} {(A ,_k P)_U} (x ,_k x_k) (x_j ,_k x_{jk}) (x_i ,_k x_{ik}) (x_{ij},_k x_{ijk}) & = 
    (\sw {i j} A x  x_j  x_i  x_{ij},_k \sw {i j} {P((x,_jx_j),_i(x_i,_j x_{ij}))} x_k  x_{jk}  x_{ik}  x_{ijk})_X & (**) \\
\end{align*}

%  The last rule might also be written for $T = (A,_k P)_U$  (CONJECTURE)
% $$
% \sw {ij} T = (\sw {ij} A,_k \sw {ij} {(x:A)→ Px})_{\SW {ij} T}
% $$
% Nope: the second part of the pair needs four functions to swap; but the first part is a single function (which also has not quite the right shape).

\begin{itemize}
\item 
(*): $p : T\param i a$ so $p \param j : (T\param i a) \param j p(j0)$. But,
the second component of the pair should be of type $(T \param j ((a ,_i p)
j0))\param i a \param j$.  That is, we need $\sw{ i j} T$ to be a conversion from
$T\param i \param j a(j0) a \param j p(j0)$ to $T \param j\param i a(j0) p(j0) a \param j$.

\item (**): The source type of the swap is
$$W = (x_{ij}:A\param i \param j x x_j x_i) \times_k P\param i \param j x x_j x_i x_{ij} x_k x_{jk}
x_{ik}$$ By computation (using other cases of the definition), all the
occurences of swap disappear in $W$.

The target type is $X$ and $X = (x_{ji}:A \param j\param i x x_i x_j) \times_k
P \param j \param i x x_i x_j x_{ji} x_k x_{ik} x_{jk}$.  One might think that swaps
should be applied to certain variables in $X$, but it is not
the case.  This can be checked by verifying that the following types
for variables match both the source and the target.

\end{itemize}
\end{definition}




\begin{definition}[Conversion]
  \begin{mathpar}
    \inferrule{t x = u} {t = λ x:A.u}

    \inferrule{t (i0) = a \\ t.i = p} {t = \CC i a p T}
  \end{mathpar}
+ congruences.
\end{definition}

 We define substitution between contexts

  \begin{align*}
    σ, δ & ::= ()  |  (σ,x=t)  |  (σ,f)
  \end{align*}

  \begin{mathpar}
    \inferrule {σ : Δ → Γ \\ f : I → J} {(σ,f) : Δ,I → Γ,J}
  \end{mathpar}
where $f$ is a function from a finite set of colours to colours or $0$.

The identity substitution $1_{Γ}$ is defined as usual and $(i0)$ is
an abbreviation for $(1_{Γ},i=0)$. Also $b[u]$ is an abbreviation for
$b(1_{Γ},x=u)$.
We define $tσ$ by induction on $t$

\begin{align*}
  Uσ &= U \\
  x(σ,f) &= xσ \\
  x(σ,y=u) &= xσ \\
  x(σ,x=u) &= u \\
  ((x:A)→ B)σ &= (y:Aσ) → B(σ,x=y) \\
  (λ x:A. b)σ &= λ y:Aσ. b(σ,x=y) \\
  (f a)σ &= fσ  (aσ) \\
  (a,_i b)_T(σ,x=u)  &= (a,_i b)_Tσ  \\
  (a,_i b)_T(σ,f)  &= (a,_i p)_Tσ & i \# f \\
  (a,_i b)_T(σ,f)  &= a(σ,f-i) & f(i) = 0 \\
  (a,_i b)_T(σ,f)  &= (a(σ,f-i),_j b(σ,f-i))_{T(σ,f)} & f(i) = j
\end{align*}

\begin{lemma}\label{lem:term-swap}
  for any $a:T$, $a.j.i = \sw {ij} T a(i0)(j0)~a(i0).j~a(j0).i~a.i.j$.
\end{lemma}
\begin{proof}
  \begin{align*}
    a &= a & \text{refl} \\
    a &= \CC i {a(i0)} {a.i} T & \text{conversion rule} \\
    a.j & = \CC i {a(i0)} {a.i} T .j & \text{congruence} \\
    a.j & = \CC i {a(i0).j} {\sw{ij} T a(i0)(j0)~a(i0).j~a(j0).i~a.i.j} T & \text{reduction} \\
    a.j.i & = \CC i {a(i0).j} {\sw{ij} T a(i0)(j0)~a(i0).j~a(j0).i~a.i.j} T.i & \text{congruence} \\
    a.j.i & = \sw{ij} T a(i0)(j0)~a(i0).j~a(j0).i~a.i.j & \text{subst} \\
\end{align*}
\end{proof}
\subsection{Swap and symmetric groups.}

The function $\sw {ij} T$ has four arguments, but the one which
matters is the last one.  Indeed, the purpose of the term $\sw {ij} T$
is to convert from $T.i.j$ to $T.j.i$, that is, to permute the role of
$i$ and $j$. In the rest of the section we ignore the first three
arguments of $\sw {ij} T$.

\begin{theorem}
For any permutation $σ$, conversion from $T.i_0 … i_n$ to $T.σ(i_0) …
σ(i_n)$ is definable.

\end{theorem}

\begin{proof}
As examples, consider the following cases:
\begin{itemize}
\item $\sw{ij} {T.k~x}$ converts from $T.k.i.j$ to $T.k.j.i$.
\item $\sw{ij} {T}.k$ converts from $T.i.j.k$ to $T.j.i.k$.
\end{itemize}

Let us see now the general case.
Let $s = \sw{i_m~i_{m+1}} {T.i_0⋯T_{m-1} x_1 … x_{2^{m-1}}}.i_{m+2}⋯i_n$.

The effect of $s$ is to convert from $T.i_0 ⋯ i_n$ to
$$T.i_0⋯i_{m-1}.i_{m+1}.i_m .i_{m+2}⋯i_n$$

That is, $s$ performs the permutation of colors $(i_m~i_{m+1})$ over
the domain $i_0 … i_n$. The permutations generated by terms in the
above form are generators of the symmetric group over $i_0 … i_n$.
\end{proof}

In the following, we write $\sw {σ} T$ the swapping term constructed
above for the permutation $σ$. The function $\sw {σ} T$ has $2^{|σ|}$
arguments, of which only the last one matters.

\begin{lemma}
  $\sw{ij} T x x_j x_i (\sw {ji} T x x_j x_i x_{ji}) = x_{ji}$
\end{lemma}
\begin{proof}
  Let $a = \CC j {\CC i x {x_i} {T(j0)}} {\CC i {x_j} {x_{ji}} {T.i}} T$.
  By Lem. \ref{lem:term-swap}, $a.j.i = \sw {ji} T x x_j x_i
  a.j.i$. By performing reductions, we obtain our result.
\end{proof}
\begin{lemma}[$σ_i^2 = 1$]
  We obtain this from first moving the $.k$'s out; then the above lemma, then pushing them back in.
\end{lemma}
\begin{lemma}[$σ_i ∘ σ_j = σ_j ∘ σ_i$ if $abs(j-i) > 1$]
  
\end{lemma}
\begin{proof}
  A consequence of using variable names instead of positions.
\end{proof}
\begin{lemma}
  [$\sigma_i ∘\sigma_{i+1} ∘\sigma_i = \sigma_{i+1} ∘\sigma_i ∘\sigma_{i+1}$]
  $\sw{kj}T.i (\sw{ik}{T.j} (\sw{ij}{T}.k~ x_{ijk})) = \sw{ij}{T.k} (\sw{ki}T.j  (\sw{jk}{T.i} x_{ijk}))$
\end{lemma}
\begin{proof}
  Compute the lhs:
  \begin{align*}
     \sw{kj}T.i (\sw{ik}{T.j} (\sw{ij}{T}.k (a.i.j.k)) ) 
    &= \sw{kj}T.i (\sw{ik}{T.j} (\sw{ij}{T} a.i.j).k ) \\
    &= \sw{kj}T.i (\sw{ik}{T.j} (a.j.i).k ) \\
    &= \sw{kj}T.i (\sw{ik}{T.j} a.j.i.k ) \\
    &= \sw{kj}T.i  (a.j.k.i) \\
    &= (\sw{kj}T  a.j.k).i \\
    &= (a.k.j).i \\
    &= a.k.j.i 
  \end{align*}
  Computing the rhs yields the same expression.
  Let $a$ be a cube constructed such that $a.k.j.i = x_{ijk}$.
\end{proof}
\begin{theorem}
  The generators satisfy the laws of symmetric groups.
\end{theorem}
\begin{proof}
  Combination of the above lemmas.
\end{proof}

\end{document}
